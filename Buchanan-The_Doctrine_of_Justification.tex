% Options for packages loaded elsewhere
\PassOptionsToPackage{unicode}{hyperref}
\PassOptionsToPackage{hyphens}{url}
%
\documentclass[
]{book}
\usepackage{lmodern}
\usepackage{amssymb,amsmath}
\usepackage{ifxetex,ifluatex}
\ifnum 0\ifxetex 1\fi\ifluatex 1\fi=0 % if pdftex
  \usepackage[T1]{fontenc}
  \usepackage[utf8]{inputenc}
  \usepackage{textcomp} % provide euro and other symbols
\else % if luatex or xetex
  \usepackage{unicode-math}
  \defaultfontfeatures{Scale=MatchLowercase}
  \defaultfontfeatures[\rmfamily]{Ligatures=TeX,Scale=1}
\fi
% Use upquote if available, for straight quotes in verbatim environments
\IfFileExists{upquote.sty}{\usepackage{upquote}}{}
\IfFileExists{microtype.sty}{% use microtype if available
  \usepackage[]{microtype}
  \UseMicrotypeSet[protrusion]{basicmath} % disable protrusion for tt fonts
}{}
\makeatletter
\@ifundefined{KOMAClassName}{% if non-KOMA class
  \IfFileExists{parskip.sty}{%
    \usepackage{parskip}
  }{% else
    \setlength{\parindent}{0pt}
    \setlength{\parskip}{6pt plus 2pt minus 1pt}}
}{% if KOMA class
  \KOMAoptions{parskip=half}}
\makeatother
\usepackage{xcolor}
\IfFileExists{xurl.sty}{\usepackage{xurl}}{} % add URL line breaks if available
\IfFileExists{bookmark.sty}{\usepackage{bookmark}}{\usepackage{hyperref}}
\hypersetup{
  pdftitle={The Doctrine of Justification},
  pdfauthor={James Buchanan},
  hidelinks,
  pdfcreator={LaTeX via pandoc}}
\urlstyle{same} % disable monospaced font for URLs
\usepackage{longtable,booktabs}
% Correct order of tables after \paragraph or \subparagraph
\usepackage{etoolbox}
\makeatletter
\patchcmd\longtable{\par}{\if@noskipsec\mbox{}\fi\par}{}{}
\makeatother
% Allow footnotes in longtable head/foot
\IfFileExists{footnotehyper.sty}{\usepackage{footnotehyper}}{\usepackage{footnote}}
\makesavenoteenv{longtable}
\usepackage{graphicx}
\makeatletter
\def\maxwidth{\ifdim\Gin@nat@width>\linewidth\linewidth\else\Gin@nat@width\fi}
\def\maxheight{\ifdim\Gin@nat@height>\textheight\textheight\else\Gin@nat@height\fi}
\makeatother
% Scale images if necessary, so that they will not overflow the page
% margins by default, and it is still possible to overwrite the defaults
% using explicit options in \includegraphics[width, height, ...]{}
\setkeys{Gin}{width=\maxwidth,height=\maxheight,keepaspectratio}
% Set default figure placement to htbp
\makeatletter
\def\fps@figure{htbp}
\makeatother
\setlength{\emergencystretch}{3em} % prevent overfull lines
\providecommand{\tightlist}{%
  \setlength{\itemsep}{0pt}\setlength{\parskip}{0pt}}
\setcounter{secnumdepth}{5}
% DEFINE PHYSICAL DOCUMENT SETTINGS HD
% media settings
\usepackage[paperwidth=5.5in, paperheight=8in]{geometry}

\usepackage{booktabs}
\usepackage{amsthm}
\makeatletter
\def\thm@space@setup{%
  \thm@preskip=8pt plus 2pt minus 4pt
  \thm@postskip=\thm@preskip
}

\usepackage{titling}
\usepackage{pdfpages}
\IfFileExists{./cover.pdf}{
  \newcommand{\myCover}{./cover.pdf}}
  {\IfFileExists{./cover.jpg}{
    \newcommand{\myCover}{./cover.jpg}}
    {\IfFileExists{./cover.png}{
      \newcommand{\myCover}{./cover.png}}{}
    }
  }
\@ifundefined{myCover}
{}
{
\pretitle{\begin{center}\includepdf{\myCover}}
\posttitle{\end{center}\setcounter{page}{0}}
\usepackage{atbegshi}% http://ctan.org/pkg/atbegshi
\AtBeginDocument{\AtBeginShipoutNext{\AtBeginShipoutDiscard}}
}
\clearpage\pagenumbering{roman}

\newenvironment{poetry}[0]{\par\leftskip=2em\rightskip=2em}{\par\medskip}

\newfontfamily\greekfont[Script=Greek]{LiberationSerif}

\makeatother

\frontmatter
\ifluatex
  \usepackage{selnolig}  % disable illegal ligatures
\fi
\usepackage[]{natbib}
\bibliographystyle{plainnat}

\title{The Doctrine of Justification}
\author{James Buchanan}
\date{2020}

\begin{document}
\maketitle

\mainmatter
\pagenumbering{roman}

{
\setcounter{tocdepth}{1}
\tableofcontents
}
\hypertarget{about-this-book}{%
\chapter*{About this book}\label{about-this-book}}
\addcontentsline{toc}{chapter}{About this book}

Republished by \href{https://classics.warhornmedia.com/}{Warhorn Classics}

Making classic Christian content available for free online in high quality, readable formats.

The latest version of this book can always be found \href{https://warhornmedia.github.io/warhorn-classics-book-template/}{here} in many electronic formats for your reading convenience on any device.

\hypertarget{downloads}{%
\subsubsection*{Downloads}\label{downloads}}
\addcontentsline{toc}{subsubsection}{Downloads}

\href{https://warhornmedia.github.io/warhorn-classics-book-template//Buchanan-The_Doctrine_of_Justification.pdf}{Download PDF}

\href{https://warhornmedia.github.io/warhorn-classics-book-template//Buchanan-The_Doctrine_of_Justification.epub}{Download ePub}

We hope this book is a blessing to you. If it is, please \href{https://warhornmedia.com/give}{make a one-time or recurring contribution} right now, sponsor a book from our upcoming list, or volunteer your proofreading or technical skills to help produce more content. Contact \href{mailto:lucas@beggarsborn.com}{Lucas Weeks} to get involved.

God bless,

---The Warhorn Team

\clearpage
\setcounter{page}{1}\pagenumbering{arabic}

\hypertarget{introduction}{%
\chapter{Introduction}\label{introduction}}

It may be thought by some that the subject of Justification is trite and exhausted; that, as one of the 'commonplaces' of Theology, it was conclusively determined and settled at the era of the Reformation; and that nothing new or interesting can now be introduced into the discussion of it. It is not necessary to say in reply to this, as some might be disposed to say, that 'what is new in Theology is not true, and what is true is not new;' for we believe, and are warranted by the whole history of the Church in believing, that Theology, like every other science, is progressive,---progressive, not in the sense of adding anything to the truth once for all revealed in the inspired Word, but in the way of eliciting and unfolding what has always been contained in it,---of bringing out one lesson after another, and placing each of them in a clearer and stronger light,---and discovering the connection, interdependency, and harmony, of all the constituent parts of the marvellous scheme of Revelation. In this sense, Science and Theology are both progressive, the one in the study of God's works, the other in the study of God's Word; and as human Science has not yet exhausted the volume of Nature, or reached the limit of possible discovery in regard to it, much less has human Theology fathomed the depths of Scripture, or left nothing to reward further inquiry into 'the manifold wisdom of God.' There may be room, therefore, for something new, if not in the substance, yet in the treatment, even of the great doctrine of Justification,---in the exposition of its scriptural meaning, and in the method of adducing, arranging, and applying the array of its scriptural proofs.

But apart from this, and looking to the character of our current literature, may it not be said that, to a large class of minds in the present age, nothing could well be more \emph{new} than the \emph{old} Theology of the Reformation? The Gospel is older than Luther; but, to every succeeding generation, it is still new,---good news from God,---as fresh now as when it first sprung from the fountain of Inspiration. It was new to ourselves,---surprising, startling, and affecting us strangely, as if it were almost too good to be true,---when it first shone, like a beam of heaven's own light, into our dark and troubled spirits, and shed abroad 'a peace which passeth all understanding.' It will be equally new to our children, and our children's children, when they come to know that they have sins to be forgiven, and souls to be saved; and to the last sinner who is convinced and converted on the earth, it will still be as 'good tidings from a far country,'---as 'cold water to a thirsty soul.' It can never become old or obsolete, for this obvious reason, that while it is 'the everlasting Gospel,' and, as such, like its Author, unchangeable,---'the same yesterday, and to-day, and for ever,'---yet it comes into contact, in every succeeding age, with new minds, who are ignorant of it, but need it, and can find no peace without it; and when they receive it as 'a faithful saying, and worthy of all acceptation, that Christ came into the world to save sinners,' they will learn from their own experience that the old truth is still the germ of 'a new creation'---the spring of a new life, a new peace, a new hope, a new spiritual existence, to which they were utter strangers before.

There are many, even in Protestant communities, who have long been familiar with the sound of the Gospel, to whom this inward sense of it, in its application to their own souls, would be nothing less than a new spiritual revelation. The doctrine of Justification, by grace, through faith in Christ, is the old doctrine of the Reformation, and the still older doctrine of the Gospel; yet the vivid apprehension of its meaning, and the cordial reception of its truth, must be a new thing in the experience of every one, when he is first enabled to realize and to believe it. The free pardon of all sin, and a sure title to eternal life, conferred by the mere grace of God, and resting solely on the redemption and righteousness of the Lord Jesus Christ,---this, as the actual and immediate privilege of every sinner, on the instant when he begins to rely on Christ alone for salvation, as He is offered to \emph{him} individually in the Gospel,---may come home, with all the freshness of new truth, even to many who bear the Christian name; and a realizing sense of them, in the conscious experience of their own souls, will be the best safeguard against the prevailing errors of the times, and the danger to which so many are at this moment exposed, of being 'tossed about with every wind of doctrine.'

If we take a calm survey of the state of religious sentiment in the present crisis,---for it is a crisis, and a very solemn one,---we can hardly fail to observe, that the minds of many are uneasy and unsettled; that there is a wide-spread feeling of unrest and dissatisfaction; and that this feeling manifests itself mainly in two apparently opposite tendencies, which have been so strikingly developed in the present age as to constitute its most marked and characteristic features;---the one is the tendency towards Rationalism, whose final goal is a cheerless and dreary Scepticism; the other, the tendency towards Ritualism, which can only find its complete realization in the Church of Rome. We see one large class of educated men relinquishing some of the most fundamental articles of the Christian faith, as if they had no need of them for their salvation, and contenting themselves with such lessons as Reason can learn by the mere light of Nature, or at least prove by rational arguments; and we see another large class of educated men betaking themselves to forms and ceremonies, to sacramental grace and ascetic practices, to auricular confession and priestly absolution, as if they could not find, in the simple Gospel of the grace of God in Christ, enough for their soul's need, without borrowing some additions to it from the inventions of men, and even from the corruptions of Popery. Each of these tendencies is a symptom of the same radical evil---the want of true peace, and good hope through grace; for those who have listened to Christ's voice, and complied with His gracious call, 'Come unto me, and ye shall find rest unto your souls,' have an anchor, both sure and stedfast, which keeps them, amidst all the fluctuations of human opinion, from drifting with the current; and neither Scepticism nor Superstition has any charms for them. 'They have drunk of the old wine, and have no desire for the new; for they say, The old is better.' Those who yield to these opposite tendencies differ in many respects from each other; but they agree in this: they have both abandoned the old doctrine of Justification, as revealed in the Gospel, and revived at the Reformation; and that cardinal doctrine is the one truth which alone can neutralize their respective errors, just as in the times of Luther it had power to overthrow alike the speculations of the Schools, and the superstitions of the Church. They differ in being more or less convinced of sin, more or less earnest in seeking salvation, more or less sincere in professing a reverential faith in God's Word,---for the hale-hearted Rationalist contrasts unfavourably in these respects with many an anxious-minded Ritualist,---but the Gospel doctrine of Justification, expounded in all its fulness, and exhibited in connection with the great scriptural principles which it involves or implies, is the most effective instrument at once for rousing the conscience of the Rationalist out of its false security, and for relieving the conscience of the Ritualist from its slavish anxieties and fears.

The false security of the Rationalist arises, not from the knowledge and belief of Christ's Gospel, but from ignorance or disbelief in regard to the demands and sanctions of God's Law; and the doctrine of Justification, as it is taught in Scripture, is fitted to break up that false security, and to awaken every thoughtful man to a sense of his real condition in the sight of God. For, in its negative aspect, it teaches us, first of all, how we cannot be justified,---it excludes the possibility of pardon and acceptance, in the case of man fallen, on the ground of his own obedience, and insists on the necessity of a satisfaction to divine justice, such as shall be at once an adequate expression of God's infinite abhorrence of sin, and an effectual means of securing all the ends of punishment under His moral government. What the Rationalist most needs at the outset is a work of the Law on his conscience,---a clearer and more impressive apprehension of the spirituality and extent of its preceptive requirements,---a deeper sense of sin---of the fact of sin, as undeniably chargeable against himself, and, especially, of the guilt of sin, as that which exposes him to imminent and awful danger,---a realizing conviction of those threatened penalties, which are expressive of God's holy hatred of it, and His inflexible determination to punish it,---and a close and faithful application of the whole Law to himself individually, as a sinner in the sight of God, standing before His awful tribunal, and awaiting His sentence, as a righteous Judge. Without some such experience as this, he will feel little or no interest in the question of Justification, and will scarcely be able to understand what it means, or what principles are involved in it. But that doctrine, when it is scripturally stated and explained in all its fulness, is related to the Law as well as to the Gospel; and for this reason it is admirably adapted to his case, just because it brings out, and places clearly before his conscience, the great fundamental principles of man's inexcusable guilt, and God's inflexible justice; and also because, when it proceeds to unfold a scheme of grace and redemption, it never loses sight of these principles, but exhibits them, all the more impressively, as exemplified and embodied in that scheme itself, which is a divine provision for the vindication of God's Law, with a view to the free exercise of His mercy towards the guilty. Let this doctrine take effect, first of all, in its Legal aspect,---bringing the Law to bear on his conscience, convincing him of the guilt which he has incurred, and awakening a sense of the punishment which he has deserved, as a sinner in the sight of a holy and righteous God; and then, but not till then, he will be prepared to understand and appreciate it, in its Evangelical aspect, when it proclaims a free pardon, but a pardon founded on a divine propitiation,---a gracious remission, but a remission by means of a divine redemption,---a full salvation, but a salvation procured by a divine satisfaction to God's eternal justice.

The anxieties of the Ritualist, again, arise from some sense of sin, combined with a more or less earnest desire of salvation; but accompanied also with much remaining ignorance in regard to the fulness and freeness of the Gospel provision for his immediate pardon and acceptance with God, and a latent feeling that there is still something that remains to be done or suffered by himself, in the way of satisfying the justice, averting the wrath, and propitiating the favour, of his righteous Judge. He has 'a zeal for God,' but 'not according to knowledge;' and 'he goes about to establish,' at least in part, 'his own righteousness,' instead of 'submitting,' at once and altogether, 'to the righteousness of God.' Hence he has recourse to confession and penance, not merely for the mortification of sin, but for relief from a sense of unforgiven guilt; and hence, too, his zeal in almsgiving and good works, not as expressions of gratitude for grace received, but as a means of deprecating the wrath, and securing the favour, of God. There is much in his state of mind which contrasts favourably with the careless indifference of multitudes who are at ease in Zion,---who have never felt that they have sins to be forgiven, or souls to be saved,---and who are only lulled into deeper security, and case-hardened in impenitence and unbelief, by their partial knowledge even of the message of mercy in the Gospel. One must feel a deep and tender sympathy with every earnest soul, which is really convinced of its sin and danger, and struggling to obtain deliverance,---and many a Ritualist may be in this condition. What he needs is a deeper and more thorough conviction of his ruined and helpless condition as a sinner, utterly unable to expiate any of his past sins by his own sufferings, or to secure divine acceptance by anything that he either has done, or can yet do: and along with this, a clearer perception of the perfect all-sufficiency of the finished work of Christ, to secure the immediate and full justification of every sinner, on the instant when he receives and rests on Him alone for salvation. The doctrine of Justification, therefore, as it is stated and explained in Scripture, is exactly suited to his case, just as it was to that of the Jewish Ceremonialist in apostolic times, and the Romish Ritualist at the era of the Reformation; for while, in its negative aspect, it excludes from the ground of his acceptance all works, whether done after faith or before it, and thus cuts up by the roots the principle of self-righteousness in its most insidious and seductive form, it proceeds, in its positive aspect, to bring in another righteousness---emphatically called 'the righteousness of God,' and to lay it down as 'a sure foundation in Zion;'---a righteousness already wrought out,---a righteousness already accepted,---a righteousness proposed to him individually by God Himself, as the ground on which he is warranted at once to rely for his present acceptance, and his eternal welfare. As soon as he betakes himself to this ground, and begins to rest upon it alone, he will find, in his blessed experience, that it is adequate to sustain his troubled soul,---to relieve it at once from all the anxieties of unforgiven guilt,---to set it free from 'the spirit of bondage which is unto fear,'---and to impart 'joy and peace in believing;' even that 'peace which passeth all understanding'---'the very peace of God reigning in the conscience through Jesus Christ,' and that 'joy of the Lord' which will be his 'strength' in duty, and his support in trial, enabling him to 'run in the way of His commandments' when the Lord has thus 'enlarged his heart.'

It was by the doctrine of Justification by grace through faith, as by a ray of light from heaven shining into their hearts, that the Reformers, in whose souls the work of the great spiritual revival was first wrought before it took effect on the face of Europe, obtained relief from the bondage of legal fear, and entered into the liberty wherewith Christ makes His people free.\footnote{The personal experience of the Reformers throws much light on the origin, and causes, of the Reformation.

  ``The different phases of this work succeeded each other in the mind of him who was to be the instrument of it, before it was publicly accomplished in the world. The knowledge of the Reformation, as effected in the heart of Luther himself, is, in truth, the key to the Reformation of the Church. It is only by studying the work in the individual, that we can comprehend the general work.'' \emph{(D'Aubigné, History of the Reformation in Europe, 5 vols., vol.~i. p.~140.)}

  'His conscience incessantly reminded him, that religion was the one thing needful, and that his first care should be the salvation of his soul. He had learned God's hatred of sin,---he remembered the penalties that His Word denounces against the sinner,---and he asked himself tremblingly, if he were sure that he possessed the favour of God. His conscience answered, No!' \ldots{} 'One day, when he was overwhelmed with despair, an old monk entered his cell, and spoke kindly to him. Luther opened his heart to him, and acquainted him with the fears that disquieted him. The respectable old man was incapable of entering into all his doubts, as Staupitz had done; but he knew his "Credo," and he had found there something to comfort his own heart. He thought he would apply the same remedy to the young brother. Calling his attention, therefore, to the Apostles' Creed, which Luther had learnt in his early childhood at the school at Mansfeld, the old man uttered in simplicity this article,---"I believe in the forgiveness of sins." These simple words, ingenuously uttered by the pious brother at a critical moment, shed sweet consolation in the mind of Luther. "I believe," repeated he to himself on the bed of suffering, "in the remission of sins."~'---Ib. pp.~159, 187.

  'In these spiritual conflicts and inward wrestlings, how grievously he was encumbered, fighting against incredulity, error, and desperation, marvellous it is to consider, insomuch, that three days and three nights together, he lay on his bed, without meat, drink, or any sleep, labouring in soul and spirit on a certain place of St.~Paul (Rom. 3:25, 26) which was---"to show His justice,"---thinking Christ to be sent for no other end but to show forth God's justice as an executor of His law,---till at length, being answered and satisfied by the Lord touching the right meaning of these words---signifying the justice of God to be executed upon His Son, to save us from the stroke thereof,---he immediately upon the same started up from his bed, so confirmed in faith, as that nothing afterward could appal him.'---Preface to English Version of Luther's Commentary on Galatians, translated by 'certain godly learned,' 1575, p.~v.

  'His great terror was the thought of "the righteousness of God,"---by which he had been taught to understand, His inflexible severity in executing judgment against sinners. Dr.~Staupitz and the confessor explained to him, that "the righteousness of God" is not against the sinner who believes in the Lord Jesus Christ, but for him,---not against us, to condemn, but for us, to justify. "I felt very angry," he said, "at the term---'the righteousness of God;'---for, after the manner of all the teachers, I was taught to understand it in a philosophic sense---of that righteousness, by which God is just, and punisheth the guilty\ldots. At last I came to apprehend it thus---Through the Gospel is revealed the righteousness which availeth with God,---a righteousness by which God, in His mercy and compassion, justifieth us, as it is written, 'The just shall live by faith.' Straightway I felt as if I were born anew; it was as if I had found the door of paradise thrown wide open. The expression 'the righteousness of God,' which I so much hated before, became now dear and precious,---my darling and most comforting word. I see the Father---inflexible in justice, yet delighting in mercy---'just,' beyond all my terrified conscience could picture Him, He 'justifies' me a sinner."~'---Chronicles of the Schönberg-Cotta Family, pp.~159, 160;---a graphic delineation of the state of feeling which prevailed at the time of Luther.

  Many touching allusions to his personal experience occur in the writings of Luther. For example, on the subject of self-righteousness, he says, 'I have myself taught this doctrine (i.e.~"of faith, by which embracing the merits of Christ, we stand accepted before the tribunal of God") for twenty years both in my preaching and my writings; and yet the old and tenacious mire clings to me, so that I find myself wanting to come to God, bringing something in my hand, for which He should bestow His grace upon me. I cannot attain to casting myself on pure and simple faith only, and yet this is highly necessary.' Again: 'He alludes to his former views when a monk, and the desire he then felt to converse with a saint, or holy person; figuring to himself under that name a hermit, an ascetic, feeding on roots; but he had since learned, that the saint was one, who, being justified in the righteousness of Christ, went on to serve God in his proper calling,---through the Spirit to mortify the deeds of the body, and to subdue his evil affections and desires.'---Scott's Continuation of Milner's History, i. pp.~233, 239.

  'Luther became a Reformer, because, in his confessional, he had learned to know the spiritual necessities of the people; because he had compassion on the poor people, even as the Saviour had compassion upon them. It was a hearty pity for the simple and ignorant, whom he, too, saw given up to the Priests, and Pharisees, and Scribes, and cheated of the highest blessings of life; it was a deep manly sorrow over the mistaken road of salvation along which the poor misled multitude were wandering, whereby Luther was inspirited to his first half-timid attempts; whereby, as he advanced, he was strengthened to stedfast perseverance,---whereby, at length, he was raised and arrayed as the mighty champion of evangelical freedom. Luther had rushed deep into the gulf of moral corruption, which was diffused among the lay commonalty, by the Romish doctrine of Justification by works. He knew from the liveliest experience the miserable condition to which the sincerest souls, the devoutest spirits, are reduced by this doctrine. He had found an escape for himself out of this tribulation---a path leading securely to the peace of the soul---in the righteousness of faith. Therefore he could not, and would not, keep silence at that which was going on around him. The princes and priests, indeed, the learned and educated, did not need, for the most part, that he should teach them the meaning of Indulgences, but the common uneducated people urgently demanded his help. This people Luther esteemed as standing exactly on the same level---as requiring, just like all other classes, to be led to the light of a purer knowledge of salvation; he neither deemed himself too high, or the multitude too low, to devote his services to them. In this state of mind, he boldly and powerfully tore down the wall of separation, which had been built up in the course of centuries, between the clergy and the laity. The mass of the laity, who had hitherto only been considered as a helpless body, to be moulded by the priests at pleasure, and to be interceded for by the Church before God, he roused, by the doctrine of Repentance and of Justification by faith, and gave them a living principle of spiritual independence and personality, supplying them with inexhaustible materials for contemplation, in the scriptural ideas of Sin and Divine Grace; and thus, out of the despised objects of an arbitrary away, he fashioned a living organized congregation of Christians, who had become free through their faith in their Redeemer.'---Hemdeshagen, Treatise on German Protestantism. See Archdeacon Hare, 'Vindication of Luther,' p.~296.

  'His deep, irrepressible, unappeasable consciousness of sin was the primary motive of his whole public life, and of all that he did for the reformation of the Church. It was on account of this deep feeling of the inward disease in the conscience that he tore off the plasters and lenitives with which the Romish quacks were wont to lull and skin over the wounds at the surface. It was on account of this that he set his foot on the scandalous fraud of Indulgences. It was by reason of this that he saw through the utter vanity of the penances and so-called good works, by which men were idly trying to purge their consciences. He felt, as St.~Paul and Augustine felt, that the evil in man does not lie in the imperfection of his outward works, but in the corruption of his heart and will. Therefore did he insist so strongly on the frailty which clings to our very best works; and therefore did he continually urge that, if we are to be justified, it must be wholly through grace, by the righteousness of our Divine Saviour, to be received and appropriated by faith, without any admixture of the works wrought by so frail and peccable a creature.'---Archdeacon Hare, Vindication of Luther, p.~135. See also Pfizer's Life of Luther.

  The experience of Calvin was similar to that of Luther. 'The Reformation was not the fruit of abstract reasoning; it proceeded from an inward labour,---a spiritual conflict,---a victory, which the Reformers won by the sweat of their brow, or rather, of their heart\ldots. We have on a former occasion sought to discover the generative principle of the Reformation in the heart of Luther: we are now striving to discern it in the heart of Calvin.'---D'Aubigné, History of the Reformation in the Time of Calvin, vol.~i. p.~20.

  'His chamber became the theatre of struggles as fierce as those in the cell at Erfurth. Through the same tempests, both these great Reformers reached the same haven. Calvin arrived at faith by the same practical way which had led Farel and Augustine, Luther and Paul.'---Ib. i. p.~522.

  'Calvin shut himself up in his room and examined himself. "I have been taught that Thy Son has ransomed me by His death; but I have never felt in my heart the virtue of His redemption." His Popish professors spoke to him. "The highest wisdom of Christians," they said, "is to submit to the Church, and their highest dignity is the righteousness of their works." "Alas!" replied Calvin, "I am a miserable sinner." "That is true; but there is a means of obtaining mercy. It is by satisfying the justice of God. Confess your sins to a priest, and ask humbly for absolution. Blot out the memory of your offences by good works." \ldots{} Calvin went to church, fell on his knees, and confessed his sins to God's minister, asking for absolution, and humbly accepting every penance imposed upon him\ldots. "O God," he said, "I desire by my good works to blot out the remembrance of my trespasses." He performed the satisfactions prescribed by the priest; he even went beyond the task imposed upon him; and hoped that after so much labour, he would be saved. But, alas! his peace was not of long duration\ldots. "Every time I descend into the depths of my heart---every time, O God, I lift up my soul to Thy throne, extreme terror comes over me." \ldots{} His heart was troubled; it seemed to him that every word of God he found in Scripture tore off the veil, and reproached him with his trespasses. "I begin to see," he said,---"thanks to the light that has been brought me,---in what a slough of error I have hitherto been wallowing,---with how many stains I am disfigured,---and, above all, what is the eternal death that threatens me." A great trembling came over him. He paced his room, as Luther had once paced his cell at Erfurth. He uttered, he tells us, deep groans, and shed floods of tears. Terrified at the divine holiness, like a man frightened by a violent thunder-storm, he exclaimed, "O God! Thou keepest me bowed down, as if Thy bolts were falling on my head."

  'Then he fell down, exclaiming, "Poor and wretched, I throw myself on the mercy which Thou hast shown us in Christ Jesus; I enter that only harbour of Salvation." He applied to the study of Scripture, and everywhere he found Christ. "O Father," he said, "His sacrifice has appeased Thy wrath; His blood has washed away my impurities; His Cross has borne my curse; His death hath atoned for me\ldots. Thou hast placed Thy Word before me like a torch, and Thou hast touched my heart, in order that I should hold in abomination all other merits save that of Jesus." Calvin's conversion had been long and slowly ripening; and yet, in one sense, the change was instantaneous. "When I was the obstinate slave of the superstitions of Popery," he says, "and it seemed impossible to drag me out of the deep mire, God by a sudden conversion subdued me, and made my heart obedient to His Word."~'---Ib. vol.~i. pp.~525--530.{]}} It was by the fearless proclamation of the same doctrine that they were enabled to impart immediate peace and comfort to many anxious inquirers, even in the cells and cloisters of the Church of Rome, who were prepared for its reception by those convictions of sin which the Law of God had power to awaken, but which all the Ritualism of Popery could not appease. And it was mainly to the influence of this one truth, carried home to the conscience 'in demonstration of the Spirit and with power,' that they ascribed their success, under God, in sweeping away the whole host of scholastic errors and superstitious practices, by which, in the course of many preceding centuries, men had corrupted the simpler faith and worship of the primitive Church. 'At the beginning of our preaching,' says Luther, 'the doctrine of Faith had a most happy course, and down fell the Pope's pardons, purgatory, vows, masses, and such like abominations, which drew with them the ruin of all Popery\ldots.. And if all had continued, as they began, to teach and diligently urge the article of Justification---that is to say, that we are justified neither by the righteousness of the Law, nor by our own righteousness, but only by faith in Jesus Christ,---doubtless this one article, by little and little, had overthrown the whole Papacy.'\footnote{Luther on the Epistle to the Galatians, English Translation (A.D. 1575), pp.~175, 176. Another testimony, equally clear and strong, may be quoted from the same work; for although it abounds in bold, and sometimes unguarded, statements, and is neither a learned nor a critical exposition of the Epistle, yet as a popular statement of Gospel truth, delivered first in the pulpit, and designed for the instruction of his congregation at Wittemberg, it is one of the noblest and freshest utterances which ever proceeded from the heart of a Christian divine. Mr.~Ward ventured to say of it in his 'Ideal of a Christian Church' (p.~172), that 'the Commentary, considered intellectually, as a theological effort, is perhaps one of the feeblest and most worthless productions ever written;' but those who have considered Archdeacon Hare's estimate of Mr.~Ward's competency to sit in judgment upon it, will probably attach more weight to the testimony of John Bunyan, who says of it, 'I do prefer this book of M. Luther on the Galatians, excepting the Holy Bible, before all the books that ever I have seen, as most fit for a wounded conscience.'---Hare's Vindication of Luther, 2d Ed. p.~155.

  Luther sets the doctrine of Justification by the blood of Christ through faith, against all the inventions of men, in the following striking terms:---

  'These words,---"the Son of God loved me, and gave Himself for me,"---are mighty thunderings and lightnings from heaven against the righteousness of the Law, and all the works thereof\ldots. What wilt thou do, when thou hearest the Apostle say, that such an inestimable price was given for thee? Wilt thou bring thy cowl, thy shaven crown, thy chastity, thy obedience, thy poverty, thy works, thy merits? What shall all these do? Yea, what shall the law of Moses avail? What shall the works of all men, and all the sufferings of the martyrs, profit thee? What is the obedience of all the holy angels, in comparison of the Son of God delivered, and that most shamefully, even to the death of the Cross, so that there was no drop of His most precious blood but it was shed, and that for thy sins? If thou couldst rightly consider this incomparable price, thou shouldst hold as accursed all these ceremonies, vows, works, and merits, before grace and after, and throw them down all to hell. For it is an horrible blasphemy to imagine, that there is any work whereby thou shouldst presume to pacify God, since thou seest that there is nothing which is able to pacify Him, but this inestimable price, even the death and blood of the Son of God, a drop whereof is more precious than the whole world\ldots. If I through works or merits could have loved the Son of God, and so come unto Him, what needed He to deliver Himself for me? Hereby it appeareth how coldly the Papists handled, yea, how they utterly neglected, the Holy Scriptures, and the doctrine of Faith. For if they had considered but only these words, that it behoved the Son of God to be given for me, it had been impossible that so many monstrous sects should have sprung up amongst them. For Faith would by and bye have answered, Why dost thou choose this kind of life, this religion, this work? Dost thou this to please God, or to be justified thereby? Dost thou not hear, O wretched man, that the Son of God shed His blood for thee? Thus true faith in Christ would easily have withstood all manner of sects. Wherefore I say, as I have oftentimes said, that there is no remedy against sects, or power to resist them, but this only article of Christian Righteousness. If we lose this article, it is impossible for us to withstand any errors or sects\ldots. What mean they to brag so much of works and merits? If I, being a wretched man and a damned sinner, could be redeemed by any other price, what needed the Son of God to be given for me?'---Luther on the Galatians, English Translation, p.~138.

  'The Church had fallen because the great doctrine of Justification through faith in Christ had been lost. It was therefore necessary that this doctrine should be restored to her before she could arise. Whenever this fundamental truth should be restored, all the errors and devices which had usurped its place,---the train of saints, works, penances, masses, and indulgences,---would vanish. The moment the ONE Mediator, and His ONE Sacrifice, were acknowledged, all other mediators, and all other sacrifices, would disappear. "This article of Justification," says Luther to Brentius, "is that which forms the Church,---nourishes it,---builds it up,---preserves and defends it. It is the heel which crushes the serpent's head."~'---D'Aubigné, History of Reformation in Europe, 5 vols., vol.~i. p.~73.

  'When the Gospel lifted up its voice in the days of the Reformation, the people listened. It spoke to them---of God, Sin, Condemnation, Pardon, Everlasting Life,---in a word, of Christ. The human soul discovered that this was what it wanted; and was touched, captivated, and finally renewed.'---D'Aubigné, History of the Reformation in the Time of Calvin, vol.~ii. p.~399. See also p.~583.}

If the doctrine of Justification by grace through faith be, as it unquestionably is, the only sovereign and effectual antidote to each of the two great tendencies of the age,---the tendency to Rationalism, on the one hand, and the tendency to Ritualism, on the other,---the re-exposition of it, in a form adapted to the more recent phases of these prevailing errors, might be, at least, a new and seasonable application of the old truth to the most urgent wants of men's minds in the present day; and, as such, it might be both interesting and useful, even if the doctrine of the Reformation were universally acknowledged to be still the doctrine of the Protestant Church. But an additional reason for a renewed exhibition of that truth, which has heretofore been unanimously recognised as the distinctive principle of the Reformation, may be found in the fact, that, of late years, and within the ranks of Protestantism itself, it has been openly assailed, as having no place either in the formularies of the Church of England, or in the writings of the Christian Fathers, or even in the Word of God itself. When old truths are attacked with new weapons, they must be vindicated by new defences, adapted to meet the most recent forms of error; and this is pre-eminently the case, at the present day, with the cardinal doctrine of Justification. It is not denied by its recent assailants that it was the doctrine of the leading Reformers, or that it was unanimously adopted and professed by all the churches which they founded, whether Lutheran or Calvinistic, with one singular exception only---the Church of England,---which, it seems, is neither Lutheran nor Calvinistic, and, of course, not Protestant,---and yet not Popish,---but purely Catholic and Apostolical! It is now alleged that the Reformed doctrine is a 'novelty,' which was introduced for the first time in the sixteenth century, and which, for fifteen hundred years, had been unknown to Catholic Antiquity, or the Church Universal; and that the Anglican Establishment, having always adhered to a complex rule of faith, composed of the Scriptures as interpreted by the Fathers, is unlike all other Protestant churches in this---that she has never adopted or sanctioned this novelty as part of her authorized creed. What renders this 'sign of the times' all the more significant and ominous is the additional fact, that all these assaults on the cardinal doctrine of the Reformation, from whatever quarter they have proceeded, whether from Rationalists or from Ritualists,---and they have proceeded from both,---have invariably had one and the same aim and direction---a return, in substance, if not in form, to the corrupt doctrine of the Church of Rome. The views on this important subject, which are now openly avowed in many influential quarters, are not only essentially the same with those which were exploded, we had hoped, for ever at the Reformation, but they are supported by the same arguments and the same interpretations of Scripture which were then current in the Popish Church, and which all the great divines of England---such as Davenant, and Downhame, and Barlow, and Prideaux, and Hooker---combated and demolished, especially in that marvellous age of sound theological learning, the seventeenth century. Yet now Protestants have been found willing to re-furbish the weapons of Bellarmine and Osorio, and to direct them anew against the very stronghold of our faith.

Within the last thirty years, several writers of unquestionable ability and learning, belonging at the time to the United Church of England and Ireland, have come prominently forward as uncompromising opponents of the Protestant, and zealous advocates or apologists for the essential principle of the Popish, doctrine on this subject. The first in order was a layman, but with a bishop as his coadjutor---Mr.~Knox of Dublin,---at one time private secretary to Lord Castlereagh, then Viceroy of Ireland, and all along the friend and correspondent of Wilberforce, John Wesley, and Hannah More, whose 'Correspondence' during thirty years with Dr.~Jebb, Bishop of Limerick, and also his 'Remains,' derive their chief interest from the zeal with which he opposes the doctrine of a Forensic Justification, and seeks to substitute for it that of a Moral Justification by our own inherent righteousness; a doctrine which is identical, in its radical and distinctive principle, with that of the Church of Rome. A seasonable antidote to some of the errors, which were thus sought to be revived in the Protestant Church, was supplied by Dr.~O'Brien, now Bishop of Ossory, in a work on 'The Nature and Effects of Faith;' but it was directed, in the first instance, against the doctrine of Bishop Bull, which made our justification to rest on faith and works conjointly; and it was only in the second edition, published with many enlargements, after an interval of more than twenty years, that the special views of Mr.~Knox were fully examined and criticised. Another valuable work appeared by George Stanley Faber, partly prepared on his own spontaneous motion, and partly called forth by a personal appeal addressed to him by the Editor of the two concluding volumes of Knox's 'Remains,' that he should throw the shield of his authority over the new views, by bringing his great learning to bear on the establishment of the historical fact, asserted by Knox, that the doctrine of Forensic Justification was a novelty introduced by the Reformation, and that it had no place in the genuine remains of Catholic Antiquity. The appeal was responded to, but in a style which must have surprised and disappointed its too sanguine author; for Faber's answer is a thorough vindication of the Protestant doctrine, and the conclusion at which he arrives, in regard alike to the schemes of Bull, Knox, and Trent, is, that 'not a vestige of any one of them can be discovered in the writings of Ecclesiastical Antiquity,'---a conclusion which is considerably stronger, as it appears to me, than is either warranted by the facts of the case, or necessary for the vindication of Protestant truth. His statement of the Protestant doctrine, and his proof of its having been taught by \emph{some} of the Fathers, are highly-satisfactory; but his conclusion, as thus stated, is utterly untenable, and need not be adopted by any one who does not hold that the unanimous consent of the Fathers is necessary to verify any article of faith. Let any one read 'Ancient Christianity,' by Isaac Taylor, and he can scarcely fail to be convinced that much grievous error, affecting both the doctrine and the worship of the Church, had crept in before the close of the second century, and that it is to be found, mixed with many precious truths, in the writings of the most esteemed Fathers. Indeed, the germ of it existed even in the primitive Church. (2 Thess. 2:7; 1 John 4:3)

Dr.~J. H. Newman, in his 'Lectures on Justification,' refers cursorily to the treatises of O'Brien and Faber, but offers no formal reply to them, otherwise than by expounding and attempting to establish his own theory, which is substantially the same, in its fundamental principle, with those of Bull, Knox, and Trent, although it is intended to be a middle way between the Protestant and the Popish doctrines. It was ably answered by Dr.~James Bennett and others. Dr.~Newman was then a minister of the Church of England, and is now a priest of the Church of Rome. This is of itself a significant indication of the tendency of the views which he had promulgated in the 'Tracts for the Times;' and it is deeply instructive to learn this additional fact, which is expressly stated in his recent 'Apology,' that in early life, and at what he still believes to have been the period of his conversion, he came under the influence of 'a definite creed,' and 'received impressions which have never been effaced or obscured,'---that he learned his first lessons in 'the school of Calvin,'---that the writer who made the deepest impression on his mind, and to whom, he says, '(humanly speaking) I almost owe my soul,' was Thomas Scott, the commentator,---that he admired the writings of Romaine, and 'hung upon the lips of Daniel Wilson;' yet all this Evangelical, and even Calvinistic, teaching has resulted in his renouncing the Protestant, and preferring the Romish, doctrine of a sinner's acceptance in the sight of God.\footnote{The titles of the works mentioned in the text, and the editions of them which will be referred to, are the following:---
  'Remains of Alex. Knox, Esq.,' in 4 vols. 8vo, 1834.

  'Thirty Years' Correspondence between Bishop Jebb and Mr.~Knox,' 2 vols. 8vo, 1834.

  Bishop O'Brien, 'Essays on the Nature and Effects of Faith,' 2d Edition, 1862.

  Geo. Stanley Faber, 'The Primitive Doctrine of Justification,' 2d Edition, 1839.

  Dr.~J. H. Newman, 'Lectures on Justification,' 2d Edition, 1840.

  Dr.~James Bennett, 'Justification as Revealed in Scripture, in opposition to the Council of Trent, and Mr.~Newman's Lectures,' 8vo, 1840. Dr.~Bennett had previously published a volume entitled, 'The Theology of the Early Christian Church,' being the Eighth Series of the Congregational Lecture,---New edition, 1855,---which touches on the subject of Justification, pp.~118--132, and has a direct bearing on the question whether the Protestant doctrine is a novelty which arose in the sixteenth century.

  Griffith's 'Reply to Dr.~Newman's Lectures,' commended by Bishop Daniel Wilson, has not come into my hands. Bateman, 'Life of Bishop Wilson,' p.~357.

  Dr.~J. H. Newman, 'Apologia pro Vita Sua,' 1864.}

This is only one specimen, selected from among many which might be mentioned, of a process which has been going on extensively for years past, in certain circles of society, and which, whether it results in avowed Romanism, or stops short at some intermediate stage, indicates, with sufficient clearness, an uneasy restlessness of mind, arising partly from some sense of sin, but also from superficial views of men's guilt and helplessness as sinners, and partly from inadequate apprehensions of the nature, value, and efficacy of the remedy which is provided for them in the Gospel. Hence the necessity of expounding anew, in these critical times, and that, too, for the benefit of Evangelical Protestants themselves, the full meaning, and the scriptural proofs, of the cardinal doctrine of the Gospel,---the doctrine of a full and free Justification, by grace, through faith in Christ alone. It is true that the writings to which I have referred, may be confined, in the first instance, to the educated classes, and may not directly affect the great body of the Protestant community; but, not to speak of the inevitable influence which, in this age of general literature, minds of high culture will ever exercise on popular opinion, it must never be forgotten that there is a deeper and more fertile source of error on this subject than false teaching from without,---it has an ally and an accomplice within; for there is profound truth in the memorable saying of Robert Trail: 'There is not a minister that dealeth seriously with the souls of men, but he finds an Arminian scheme of justification in every unrenewed heart.'\footnote{Robert Traill (of London), 'A Vindication of the Protestant Doctrine of Justification,' Works, vol.~i. p.~321. Reprinted by the Free Church Committee on Cheap Publications.}

That these Lectures may be adapted to the exigencies of the present times, it is necessary to keep steadily in view the theories and speculations which have recently appeared, and to suggest such considerations as may serve to neutralize or counteract their injurious influence. But they are designed to be didactic, rather than controversial. For it has long been my firm conviction, that the only effective refutation of error is the establishment of truth. Truth is one, error is multiform; and truth, once firmly established, overthrows all the errors that either have been, or may yet be, opposed to it. He who exposes and expels an error, does well; but it will only return in another form, unless the truth has been so lodged in the heart as to shut it out for ever. The great object, therefore, should be, to expound the doctrine of Justification in its full meaning, as it is revealed in Scripture,---to illustrate the great principles which are involved or implied in it,---to adduce and apply the scriptural proofs on which it rests,---and to contrast it with such other methods of obtaining pardon and acceptance with God as men have devised for themselves; and this, with a view to two practical results: first, to direct some, whose consciences have been awakened but not appeased, to a sure ground of immediate pardon and acceptance; and secondly, to direct believers, who are still burdened with doubts and fears, to such views of the nature, grounds, and evidences of this great Gospel privilege, as may serve, under the divine blessing, to raise them to a more comfortable enjoyment of it, by adding the 'assurance of faith' and 'hope' to 'the assurance of understanding.'

\hypertarget{history-of-the-doctrine-of-justification}{%
\chapter{History of the Doctrine of Justification}\label{history-of-the-doctrine-of-justification}}

In the printed version, this footnote was attached to the title. But I don't know how to do that here.\footnote{Many years ago, Bishop O'Brien announced his intention to prepare a History of the Doctrine of Justification; but that intention has not yet been carried into effect, and there is scarcely any work in the English language which can be said to supply the want. It is in every respect desirable, that one so thoroughly competent for the task, in point both of ability and learning, should take up this comprehensive subject, which can only be treated cursorily in a series of Lectures like the present, and would require an entire volume for its illustration.

  The sources of information on the subject are either general or special. Some works give the history of the doctrine,---or materials for constructing its history,---in all ages, including the faith of the Church in regard to it under the Old, as well as the New, Dispensation;---others give its history, either in the Old Testament, or in post-apostolic times, only.

  To the first class belong the Scriptures of the Old and New Testaments, which must ever have the first place assigned to them, as being inspired records, both of the divine revelations which were vouchsafed from time to time to the Church, and of the faith and worship which were maintained in it from the beginning. A sound exposition of Scripture, which should follow the historical course of Revelation from its commencement to its close in the sacred canon, would be the best history of both.

  The 'Magdeburg Centuriators,'---viz., M. Flacius Illyricus, Joannes Wigandus, Matthæus Judex, Basilius Faber, and others who were associated with them,---were induced to write the History of the Church anew by the conviction, that previous historians had not given due prominence to the doctrinal truths of Scripture, especially to the doctrine of Justification; and they have collected valuable materials for its history, both under the Old dispensation and the New. Two of their number,---Joannes Wigandus and Matthæus Judex,---published separately from their great work, in 1563, a thick quarto volume, entitled, 'SYNTAGMA, seu Corpus Doctrinœ ex Veteri Testamento tantùm Collectum,' in which they collect together, under distinct heads, the great truths which are common to both Testaments; and treat 'De Evangelio,' p.~944, 'De Justificatione Peccatoris coram Deo," p.~962, 'De Fide,' p.~1003, 'De Bonis Operibus,' p.~1019, and other cognate topics. In their larger work, the Centuriators give the history of the doctrine under the New Testament dispensation, but not continuously; the passages which relate to it must be collected from the account of each century. Century I., Book i. c.~iv., includes the teaching of our Lord, pp.~9--111, and of the Apostles, pp.~219--278, 'De Justificatione Hominis coram Deo;' and the same topic is resumed in each successive century.

  The two works of Buddeus,---'Historia Ecclesiastica Veteris Testamenti,' and 'Ecclesia Apostolica,'---embrace the teaching of both Testaments. Four admirable 'Exercitations,' by Witsius, give the history of the opinions which prevailed among the Gentiles and the Jews; also the doctrine which was taught by the Apostles: 'Miscell. Sacra,' vol.~ii. pp.~668--752. They are entitled, respectively,---'De Theologia Gentilium in Negotio Justificationis,' pp.~668--697,---'De Theologia Judæorum in Negotio Justificationis,' pp.~698--721,---'De Controversiis quæ Apostolorum ætate in Ecclesia Christiana circa Justificationem ortæ sunt,' pp.~721--731,---'De Mente Pauli in Negotio Justificationis,' pp.~732--752. These dissertations were occasioned by Dr.~Cave's 'Antiquitates Apostolicæ,' on that work being translated and published on the Continent; and were designed as an answer to it. Dr.~Cave's opinion was, that the doctrine, as taught by the Apostles, excluded Justification by ceremonial observances, and left it to depend entirely on Faith; but that this Faith, which is the only condition of the New Covenant, is not any special grace, having an office or function distinct from that of other graces, but is rather comprehensive of them all; and that, therefore, works of evangelical obedience are not excluded from the ground of our acceptance with God. To this class of works may be added President Edwards' 'History of Redemption.'

  The works which have been mentioned afford materials for constructing the history of the doctrine in the Church both of the Old and New Testaments. Many other works give, more or less fully, the history of the doctrine either in the Old Testament, or in post-apostolic times. Of works on the Old Testament, we may mention, Hengstenberg's 'Christology of the Old Testament,' 4 vols. (T. and T. Clark, Edinburgh), with the older 'Christology' of Robert Fleming, jun.; and that most instructive and edifying series of Lectures, in 4 vols., entitled, 'Christ as made known to the Ancient Church,' by my late venerable colleague, Dr.~Gordon, of the High Church, Edinburgh. Of works relating to the post-apostolic History of the Doctrine, we may mention, Dr.~Hagenbach, of Basle, 'Compendium of the History of Doctrines,' vol.~ii. pp.~267--274, and 447--460; Dr.~Shedd, of America, 'History of Christian Doctrine,' Book v. 'History of Soteriology,' vol.~ii. pp.~201--386; Dr.~Muenschen, of Marpurg, 'Elements of Dogmatic History,' translated by Dr.~James Murdoch, 1830, c.~vii. pp.~72--80, and 184--190.

  Petavius does not treat of Justification as a distinct topic in his 'Dogmata Theologica' (6 vols. fol., Antwerp, 1700), but frequent references occur to it; as when he speaks of 'Preparations for Justification,' vol.~i. lib. x. c.~xxvii. s. 12,---of 'Justice,' or 'Righteousness,' vol.~i. lib. vi. c.~viii. s. 6; lib. x. c.~ii. s. 4, c.~xiv. s. 1,---of 'Justification and Adoption,' vol.~ii. lib. viii. c.~4, 5, 10, 1; in vol.~iii. 'De Pelagianis et Semipelagianis,' p.~336, and 'De Tridentini Concilii Interpretations,' and 'De Sancti Augustini Doctrina,' p.~353, when he refers to the conflicting interpretations by Soto and Vega of the Canons and Decrees of the Council, c.~xv.; and in vol.~v. vi. 'De Incarnation Verbi,' in 16 books.

  One of the most useful works on the subject is that of J. Forbes (of Corse), 'Instructiones Historico-Theologicæ.' See lib. viii. c.~2, 5--10, but especially c.~23, 24, pp.~423--429.

  Chemnitz gives 'Veterum Testimonia de Justifications' in the first part of his 'Examen Concilii Tridentini,' p.~141.

  All the general histories of the Church may be consulted, such as Dr.~Kurtz's 'History of the Old Covenant,' and Neander's, Weismann's, Mosheim's, and Milner's, Histories of the Christian Church.

  The special sources of information, in regard to the state of the doctrine at particular eras, will be referred to in connection with each of the great controversies which have arisen in regard to it. But full information cannot be obtained by merely reading an historical narative; and recourse must be had to two or three of the best writers on each side of every discussion, as it passes under review.}

\hypertarget{history-of-the-doctrine-in-the-old-testament}{%
\chapter{History of the Doctrine in the Old Testament}\label{history-of-the-doctrine-in-the-old-testament}}

By Justification we mean---man's acceptance with God, or his being regarded and treated as righteous in His sight---as the object of His favour, and not of His wrath; of His blessing, and not of His curse. This is the formal definition, or generic description of it, whether it be considered as an act on the part of God, or as a privilege on the part of man. Many have taken a partial and defective view of it, as if it consisted merely in the pardon of sin; but in the case of a moral and responsible agent, placed in a state of probation, with a view to reward or punishment, there might, and there would, have been justification, had there been no sin to be forgiven, as is evident from that of the angels who 'kept their first estate.'

When Justification is thus defined or described, it may seem to be possible only in the case of innocent and unfallen beings, and to be utterly beyond the reach of such as are guilty and depraved. And so it is on the footing of mere law, and on the ground of personal obedience to it. For that law is the rule of God's righteous judgment; and, His judgment being ever according to truth, He cannot justify the wicked, any more than He can condemn the righteous, when respect is had solely to their personal character and conduct. The law which proclaims the punishment of sin can contain no provision for the pardon of it; and if it be the sole rule by which we are to be justified or condemned, our justification is impossible; for 'our own hearts condemn us, and God is greater than our hearts, and knoweth all things.' Had we been left, therefore, to the mere light of nature, and without a supernatural revelation of 'the will of God for our salvation,' we could never have answered the question---'How shall a man be just with God?'

This is the great problem which the Gospel of Christ, and that only, has undertaken to solve; and it is the history of that problem, and of its divine solution, as contrasted with the devices and inventions of men, which we propose to trace through its successive stages, from the beginning down to the present day. But we cannot understand the relation which subsists between the Law and the Gospel, in so far as they bear respectively on the question at issue, without some knowledge of the fundamental principles which are common to both; and, for this reason, we must consider, in the first instance, the Justification of the Righteous, and thereafter proceed, in the second, to the Justification of Sinners.

I. The Justification of the Righteous comes first. The doctrine of Justification had its origin in the earliest revelations which were made to the first parents of our race in primæval times. It cannot be ranked among the truths of, what is commonly called, Natural Religion; for, although there is a valid natural evidence for the being and attributes of God, for His providential and moral government, for the responsibility of man and the immortality of his soul, such as might suggest the idea of retribution, and awaken a foreboding of future judgment; yet the tenure on which life should be held, and the terms on which the favour of God should continue to be enjoyed, could only be determined by a free act, and announced by an authoritative revelation, of His sovereign will. Viewed in the mere light of reason and conscience, the punishment of sin is far more certain than the reward of obedience; for while it is evident that, under a scheme of moral government, sin deserves punishment, it is not so clear that any obedience which man might render could, strictly speaking, merit a reward, or constitute a claim in justice to anything more than exemption from penal suffering in a state of innocence. Yet this was a subject which could hardly fail to engage the thoughtful inquiry of a rational, responsible, and immortal being, and it deeply concerned him to know what was the will of God in regard to it.

We find, accordingly, that after God had revealed Himself in the first instance, as the Creator of the world, and instituted the Sabbath as a weekly day of religious rest and worship, the next revelation which was addressed to the common father and representative of the race, was directed to this precise point, and made known the terms on which 'eternal life,'---not the continuance merely of a state of conscious personal existence, but the continuance of that holy and happy life which was enjoyed in a state of original righteousness, and which consisted essentially in the divine favour and image,---should be infallibly secured, to him, and to all his posterity, by the express promise, and the unchangeable faithfulness, of God. In that primæval revelation He made Himself known to our first parents, not only as their Creator and Benefactor, but also as their Lawgiver, Governor, and Judge; and, founding upon that Moral Law which He had already written on the fleshly tablets of their hearts, and which bound them equally to believe whatever God might say, and to do whatever God might command, He imposed upon them a single positive precept as the test of their obedience,---connecting this precept, on the one hand, with the penalty of death, and, on the other, with the promise of eternal life. The precept, the penalty, and the promise, were associated with a visible sign or symbol in the tree of life, which was the sacrament of this dispensation; and the real import of each of these must be distinctly apprehended if we would form a correct conception of the method of Justification which was thus revealed.

The precept required perfect obedience; for although it was restricted to one duty in the shape of a positive observance, that duty was enjoined as a test of man's submission to God's authority---of his faith in God's word, and his obedience to God's will---of his love to God, and his desire for the continued enjoyment of His favour and fellowship; and such a test was evidently framed on the principle that 'every sin deserves God's wrath and curse,' and that 'whosoever shall keep the whole law, and yet offend in one point, he is guilty of all.' The penalty denounced 'death' as the wages or desert of sin; not, as some have said, mere temporal death, or the dissolution of the union between body and soul; nor the annihilation of the soul, and the destruction of conscious existence, at the close of the present life,---nor even the mere natural effect of sin itself as it is a subjective evil, or as it is in its essential nature, a loathsome and mortal disease of the soul, which is destructive of spiritual life,---but the death denounced was primarily, and principally, the loss of God's 'favour, which is life, and of His loving-kindness, which is better than life,' and the infliction of penal suffering, as at once the effect and the manifestation of God's 'wrath' and 'curse' on account of sin.\footnote{These various opinions are represented respectively by the following writers:---The first by Dr.~Taylor of Norwich, in his 'Scripture Doctrine of Original Sin,' and his 'Key to the Apostolic Writings,' which are answered by President Edwards in his 'Great Christian Doctrine of Original Sin,' Works, vol.~ii. pt.~ii. sec.~ii. The second by Henry Dodwell, in his 'Epistolary Discourse, proving that the Soul is naturally Mortal, but immortalized by its union with the Divine Baptismal Spirit, imparted only by the Bishops;' which was answered by Dr.~S. Clarke in his 'Letter to Mr.~Dodwell.' It has been recently revived, in a different form, by Mr.~Edward White, in his work entitled, 'Life in Christ' (1846)---which is directed to prove that 'Immortality is the peculiar privilege of the regenerate.' The third by many modern writers, who make spiritual death to consist entirely in sin, as a subjective moral evil, and overlook the wrath and curse of God on account of past transgressions. On this subject, see the profound treatise of Dr.~Thomas Goodwin, 'An Unregenerate Man's Guiltiness before God in respect of Sin and Punishment,' Works, vol.~x. pp.~1--56, Nichol's Edition.} The promise,---which was implied in death being threatened only in the event of transgression, and which was visibly embodied and symbolized in the 'tree of life,'---secured, not merely the continuance of temporal life, nor even a state of immortal existence, but the perpetuity of that holy blessedness which consisted in the favour and fellowship of God; for the life, which was promised, was the counterpart of the death, which was threatened; and these are identified with God's blessing, and God's curse: 'Behold! I set before you this day a blessing and a curse; a blessing, if ye obey the commandments of the Lord your God, \ldots. and a curse, if ye will not obey the commandments of the Lord your God.' God's 'blessing,' and God's 'curse,'---the one as comprehending all the good, and the other as comprehending all the evil, which flow from them respectively,---these were the sanctions of God's law. The benefits bestowed, and the penalties inflicted, are only effects or manifestations of God's favour, which is life, or of God's curse, which is death.

The Law, thus promulgated, became a divine covenant, in which God was pleased to bind Himself by His promise, and to become, as Boston says, 'debtor to His own faithfulness' for its fulfilment,---while He bound the father of the human family, as the divinely appointed Representative and Federal Head of his posterity, by the obligation of the precept, on peril of penal condemnation in the event of disobedience. There was much grace in this covenant; for eternal life could never have been earned, or claimed as due, on the ground of merit at the hand of justice, however perfect man's obedience might be to the precept of the Law, while now, in virtue of the free and unchangeable promise, it might be claimed on the ground of God's faithfulness and truth; and further, the precept itself, connected as it was with a solemn penalty, was yet of a 'protective character;' for while it did not exclude the possibility of sin, which seems to be necessarily involved in a state of probation and trial, it narrowed the range of man's danger by summing up his duty in one positive precept as the test of his obedience to the whole Law, and making him invulnerable at all other points as long as he remained stedfast in submitting to the only restriction which had been imposed on his freedom.\footnote{Professor M'Laggan's Lectures, pp.~307--367.} Yet while it had much grace in it, this Law is properly called a Covenant of Works; for it established a certain relation between obedience and reward, such as that which subsists between work and wages. Eternal life was promised on condition of obedience, and, on that condition being fulfilled, the reward might have been claimed, not as a 'reward of grace,' but 'of debt.' Even then it could not have been claimed on the ground of merit, as if it were due in justice to our obedience, but it might have been claimed on the plea of covenant faithfulness, and that, too, on the ground of personal obedience.\footnote{Rom. 4:4: 'μισθος κατὰ χάριν,---μισθὸς κατὰ τὸ ὀφείλημα.' 'Meritum ex condigno' is distinguished, even by Popish writers, from 'Meritum ex pacto' or 'ex promissione;' but in treating of the latter, in connection with the rewards which are promised to believers under the New Covenant, they overlook the fact that these are promised on account of the merits of Christ. There is still a wide difference between 'rewards of debt,' and 'rewards of grace;' for while both were promised,---the one under the first, the other under the second, covenant,---yet the former were to be bestowed on the ground of personal obedience, while the latter are bestowed on account of the obedience of Him with whom the covenant was made on behalf of His people; that is, on the ground of His vicarious and imputed righteousness. 'The whole tenor of Revelation shows, that there are but two methods whereby any of the human race can be justified: either by a perfect obedience to the law in their own persons, and then "the reward is of debt," i.e.~pactional debt, founded on the obligation of the covenant, not springing from any worth in the obedience. Or else, because the Surety of a better covenant has satisfied all demands in their stead; and then "the reward is of grace," Rom. 4:4.'---Hervey's Works, vol.~ii. p.~296.}

Such was the first method of Justification. The Law, in its covenant form, was 'ordained unto life;' and its terms were simply these, 'This do, and thou shalt live,' but 'the soul that sinneth, it shall die.' The Law provided for the justification of the righteous, and of the righteous only. It was evidently adapted to the case of man while he was yet, not only innocent and sinless, but possessed of original righteousness, enjoying the 'favour of God, which is life,' and retaining that divine 'image' in which he was created. But the favour of God was forfeited, and the death of the soul incurred, by sin. There was something now which 'the Law could not do, in that it was weak,' not in itself, but 'through the flesh,' or the fallen state of man: it could no longer give life, simply because righteousness could not come by a law which had been broken,---and although it still remains in force, it is only as 'a ministration of death,' a 'ministration of condemnation.'2 For this reason, no sooner had man transgressed the precept, than he was solemnly debarred from the sacrament, of this covenant;---he was shut out from Eden, and God fenced it round with 'cherubim, and a flaming sword which turned every way, to keep the way of the tree of life.'

The Law, as it was promulgated in a state of holy innocence, while man still retained the 'image and likeness' of God, was adapted to his powers as an unfallen being, and related only to the justification of the righteous. It made no provision, and, from its very nature, it could make none, for the acceptance of sinners. It is a method of justification by Law; and Law, as such, when it is applied in judgment, must either justify or condemn. But there are many reasons why the Law, which justifies the righteous only, and condemns every sinner, should be carefully studied, in the first instance, in order that we may be prepared to understand and appreciate that other method of justification which the Gospel reveals. The Law and the Gospel are so related, that the one presupposes the other, and is founded upon it; and, by a marvellous device of divine wisdom, the justification of sinners is brought into intimate connection with that same Law, by which they are convicted and condemned. The Law worketh 'wrath,' the Gospel proclaims 'reconciliation;' but the two are connected by means of a 'redemption,' wrought out by One who 'redeemed us from the curse of the Law, by being made a curse for us.' The penalty of the Law takes effect, not on the sinner, but on a Divine Substitute; and the end of punishment being thus secured, pardon is proclaimed on the ground of a propitiation. But this method of justification for sinners, although it be 'without the Law,' as being above and beyond what the mere Law could provide, is so closely related both to its preceptive and penal requirements, that we can form no scriptural views of the one without some suitable conception of the other. Hence the careful study of the Law, as a covenant of works, is necessary at all times to the right understanding of the Gospel, as a covenant of grace: and it is peculiarly seasonable in the present age, when the eternal Law of God is supposed, by some, to have been abrogated, and, by others, to have been modified or relaxed. We must believe that the Law of God, in all its spirituality and extent, is still binding, if we are to feel our need of the Gospel of Christ; and we must be brought to tremble under 'the revelation of wrath,' if we are ever to obtain relief and comfort from 'the revelation of righteousness.'\footnote{On the first covenant of life, see Witsius, 'De Œconomia Fœderum Dei,' lib. i. c.~ii.--viii. pp.~8--99; Burmann, 'Synopsis,' vol.~i. lib. ii. c.~ii. pp.~389--475; Bishop Hopkins on 'The Two Covenants;' Boston on 'The Covenant of Works;' Dr.~Russel (of Dundee) on 'The Adamic and Mediatorial Dispensations'; Dr.~Meikle (of Beith) on 'The Edenic Dispensation;' Mr.~Strong on 'The Covenants;' Mr.~Barrett on 'The Covenants,' pp.~38--75; and many more. As some have denied the literal truth of the Mosaic narrative on this subject, see also Holden's 'Dissertation on the Fall of Man, in which the literal sense of the Mosaic Account of that event is Asserted and Vindicated,' 1823; also Jo. Witty, 'Vindication of the History of the Fall of Adam,' 1705.

  'I begin with the first revelation which God made of Himself, and of His will, to man in the beginning of time; and from thence 'I would descend to later revelations, both before, and in, Gospel times. The holy, all-wise God, having created reasonable creatures, gave to them a Law, the rule of that obedience and duty which is the natural result of the relation between God the Creator, and such creatures. This Law required perfect sinless obedience. No less could God call for; no less was suited to the state of innocence and perfection, wherein man was created. This Law, given at first, was written on the heart, and needed not to be externally proposed. That positive prohibition, Not to eat of the tree of the knowledge of good and evil, was but for the trial of obedience; and the tree itself, a sacrament or symbol of death, in case of disobedience, as the tree of life was a symbol or sacrament of life, in case of obedience. These symbols clearly show that the Law was established into a covenant. And a covenant it was, truly and properly; for Adam had no right to deny his consent to the terms which God proposed; and, being yet sinless and holy, he had no will thereto, but agreed both to the preceptive part, and to the sanction, as "holy, just, and good."~'---Beart, Vindication of the Eternal Law and Everlasting Gospel, p.~2. London, 1753. This work is recommended by Hervey ('Theron and Aspasio,' vol.~ii. p.~20) as a 'most excellent treatise,' which has 'the very sinews of the argument, and, the very marrow of the doctrine.' It consists of two parts, and has been frequently reprinted.}

II. The doctrine of the Justification of Sinners had its origin immediately after the Fall. Having broken the condition of the covenant, by an act of wilful transgression, our first parents had incurred the double guilt, of disbelieving God's word, and of disobeying God's will. They had thereby forfeited the promise of life, and incurred the penalty of death. They had listened to the tempter, first, when he suggested a doubt as to the divine prohibition, and again, when he denied the certain execution of the divine penalty; but now they were undeceived by their own conscious experience; for, no sooner had they committed sin, than immediately conscience awoke as God's vicegerent in their own breasts, and they were self-convicted and self-condemned. That one act had changed their whole relation to God, and reversed, at the same time, all their feelings towards Him; they had forfeited His favour, and incurred His wrath; and instead of being, as He once was, the object of their supreme love and confidence, He had become the object of their jealousy, suspicion, and distrust. A sense of His displeasure produced, through fear, a feeling of enmity; and that enmity could never have been subdued, without some token of His continued interest in their welfare, and of His disposition to receive them again into His favour. So sudden and so great had been the change which sin had wrought in all their relations and feelings towards Him, that they were ashamed, and afraid, and would have hid themselves, if they could, 'from the presence of the Lord God.' They now dreaded the penalty, because they felt it to be deserved; and they dreaded it, not merely on account of the sufferings which it might entail, but also, and chiefly, as it was an expression of God's displeasure, and a manifestation of His wrath.

When they were summoned to appear before Him as their Judge, they must have been prepared to hear---what alone the Law could have led them to expect---a sentence of condemnation. But He was pleased to interpose at this critical moment for their immediate and effectual relief. He pronounced, in their hearing, a curse on 'the serpent and his seed,' and conveyed, in the very bosom of that curse, an intimation of His sovereign purpose of grace and mercy towards themselves. There was a profound significance in this brief and simple, but most comprehensive, statement of God's purpose, when viewed in connection with the circumstances in which they were then placed, and the convictions which had been already awakened in their minds. It implied that God, instead of appearing against them as their enemy, was to interpose for them as their friend; that He had formed a purpose of grace and mercy towards them, and had devised a plan for their relief and restoration. It implied that, with a view to their ultimate deliverance, they were to be spared, and placed under a dispensation of forbearance, during which the execution of His penal sentence should be suspended; for their 'seed' is distinctly mentioned, intimating that their lives were to be prolonged. It implied that, in the exercise of His sovereignty, He had taken their case entirely into His own hands, as if He, and He only, had the right, and the power, to deal with it: 'I WILL PUT enmity between thee and the woman, and between thy seed and her seed;'---words which clearly intimate that the whole plan of their deliverance originated in His sovereign purpose, and that it was to be accomplished by His own agency. It implied that His purpose of mercy towards them should be effected, not immediately and directly, by a mere act of indemnity as an expression of His sovereign will, or by the direct exertion of His almighty power, but through the mediation of 'the Seed of the woman,' who should be born into the world, and enter into conflict with Satan, so as to be himself a sufferer, yet to come off victorious. It implied that, through this human deliverer, God would break up the unholy league which had been formed betwixt them and that evil spirit,---emancipating them from his usurped dominion, crushing his power, frustrating his schemes, and destroying his works. It implied that their salvation was secured by a purpose of grace which was absolute, as it depended on the mere 'good pleasure of His will,' and by a promise which was unconditional, since no terms are imposed, and no works required, and no mention made of any human agency, excepting only the sufferings and work of the 'woman's Seed.' It implied that the 'woman's Seed'---the promised deliverer---was now to be the Hope of the world, and the Head of a redeemed people, who should be rescued from the curse of the Law, and restored to the favour and friendship of God; for Adam, the head of the old covenant, is superseded under the new, by One who is predicted and promised as 'the Seed of the woman.' It implied an 'election according to grace,' for distinct mention is made of 'the woman's seed,' and 'the serpent's seed;' and the serpent's seed are left under the curse, while the woman's seed are delivered from it. And it points forward to a mysterious conflict between Satan and the promised Saviour, in which there should be mutual 'enmity' and 'bruising,'---opposition and suffering on both sides,---but resulting in victory over the Wicked One.

The announcement of God's purpose of mercy was made in general terms, and it gave no definite information on many points which are now more fully and clearly revealed; but it contained enough to lay a solid foundation for faith and hope towards God, and it was the first beam of Gospel light which dawned on our fallen world. For what is the Gospel, if it be not the revelation of God as 'the just God and the Saviour,'---reconciling sinners to Himself by a Redeemer,---not imputing their trespasses unto them, but accepting them as righteous, admitting them to His favour and fellowship, and giving them peace of conscience here, and the hope of eternal life hereafter, by faith in His gracious promise? God had already revealed Himself as the Lawgiver, Governor, and Judge; He now reveals Himself as the 'just God and the Saviour;'---as the just God, for He pronounces a curse on the serpent, and predicts the sufferings also of the woman's Seed, thus manifesting His holy displeasure against sin; and yet as 'the Saviour,' for He promises a Deliverer, who should suffer indeed on account of sin, but, by suffering, accomplish the salvation of sinners. Looking to God in this character, our first parents might believe, as Abraham afterwards believed, in 'Him that justifieth the ungodly;' and looking to the promised 'Seed,' they might believe, as Abraham afterwards believed, that in this Seed should 'all the families of the earth be blessed.' The object of faith in these primitive times was, in substance, the same as now: God in His revealed character as 'just, and the justifier of him that believeth;'---with this difference, that the Saviour was then promised as 'coming,' but is now proclaimed as 'having come.'\footnote{The first promise, or primeval Gospel. 'De Evangelio; Quid sit. Evangelium est doctrina à Deo immediatè patefacta, de gratuita reconciliatione hominum lapsorum, et remissione peccatorum per Messiam, quæ fide accipienda est, adferens atque impertiens justitiam coram Deo, Messiæ passions acquisitam, pacem conscientiæ, et vitam eternam. Hæc definitio ex suavissimis dictis Scripturæ sacræ---Gen.~3:22, et aliis sumpta est.'---Wigandus and Judex, Syntagma, p.~944.

  The effect of this revelation of God's purpose of mercy in changing the whole state and experience of our first parents, is stated, with a grand simplicity, by John Knox, when, speaking of the three cardinal points,---our sin and misery,---God's promise of grace,---and the effect of faith in it,---he says, 'All this plainly may be perceived in the life of our first parent Adam, who, by transgression of God's commandment, fell in great trouble and affliction,---from which he should never have been released, without the goodness of God had first called him. And, secondly, made unto him the promise of his salvation, the which Adam believing, before ever he wrought good works, was reputed just. After, during all his life, he continued in good works, striving contrary to Satan, the world, and his own flesh.'---Knox's Works, vol.~iii. p.~439,---the admirable edition, for which the Church is indebted to David Laing, Esq., of the Library of the Writers to the Signet, Edinburgh.

  'Had Adam felt,' says Zuingle, 'that he had anything remaining after his fall which might gain the favour of his Maker, he would not have fled "to hide himself;" but his case appeared to himself so desperate, that we do not read even of his having recourse to supplication. He dared not at all to appear before God. But here the mercy and kindness of the Most High are displayed, who recalls the fugitive, even when, with a traitor's mind, he is passing over to the camp of the enemy, and not even offering a prayer for pardon; receives him to His mercy; and, as far as His justice would permit, restores him to a happy state. Here the Almighty exhibited a splendid example of what He would do for the whole race of Adam, sparing them, and treating them with kindness, even when they deserved only punishment. Here, then, Religion took its rise, when God recalled despairing, fugitive man to Himself.'---Zuingle, De Vera et Falsâ Religione, p.~169.

  'All the promises,' says Luther, 'are to be referred to that first promise concerning Christ, "The seed of the woman shall bruise the serpent's head," Gen.~3:15. So did all the prophets both understand it, and teach it. By this we may see that the faith of our fathers in the Old Testament, and ours now in the New, is all one, although they differ as touching their outward object. Which thing Peter witnesseth in the Acts (15:11): "We believe that, through the grace of the Lord Jesus Christ, we shall be saved, even as they." \ldots{} The faith of the fathers was grounded on Christ which was to come, as ours is on Christ which is now come. Abraham in his time was justified by faith in Christ to come; but if he lived at this day, he would be justified by faith in Christ now revealed and present. Like as I have said before of Cornelius, who at the first believed in Christ to come, but, being instructed by Peter, he believed that Christ was already come. Therefore the diversity of times never changeth faith, nor the Holy Ghost, nor the gifts thereof. For there hath been, is, and ever shall be, one mind, one judgment and understanding, concerning Christ, as well in the ancient fathers, as in the faithful which are at this day, and shall come hereafter. So we also have a Christ to come; and to believe in Him, as the fathers in the Old Testament had. For we look for Him to come again in the last day with glory, to judge both the quick and the dead, whom now we believe to be come already for our salvation.'---On the Galatians, pp.~187, 188. 'All the faithful have had alway one and the self-same Gospel from the beginning of the world, and by that they were saved.' \ldots{} 'Christ came in spirit to the fathers of the Old Testament, before He came in the flesh. They had Christ in spirit. They believed in Christ which should be revealed, as we believe in Christ which is now revealed, and were saved by Him as we are, according to that saying, "Jesus Christ, the same yesterday, and to-day, and for ever." "Yesterday," before the time of His coming in the flesh; "to-day," when He was revealed "in the time before appointed." Now and "for ever" He is one and the same Christ: for even by Him only, and alone, all the faithful which either have been, be, or shall be, are delivered from the law, justified, and saved,---Ibid. pp.~258, 295.}

Such are some of the truths which are expressed or implied in the first promise of a Saviour, as it was conveyed in a curse pronounced against the serpent. They were fitted to produce a feeling of reverence for the justice of God, as the supreme Lawgiver, Governor, and Judge, both of men and of higher orders of invisible beings; and yet also a feeling of hope and trust in His mercy, through that Saviour whom He had promised to raise up for their deliverance.\footnote{In the question respecting the Justification of Old Testament believers, the principal points are these,---the fact that they were justified,---the reason or ground of their pardon and acceptance,---and the means by which they were made partakers of this privilege.

  The fact that they were justified, in the full Gospel sense of that expression, can scarcely be questioned; since they are expressly declared to have been freely forgiven, and restored to the favour and friendship of God. The fact was even divinely attested: Abel 'obtained witness that he was righteous;' Enoch, 'before his translation, had this testimony, that he pleased God' (Heb. 11:4, 5). They not only possessed, but they enjoyed, this Gospel privilege; for 'David describeth the blessedness of the man unto whom God imputeth righteousness without works, saying, Blessed are they whose iniquities are forgiven, and whose sins are covered; blessed is the man to whom the Lord will not impute sin' (Rom. 4:6, 7; Ps. 32). 'I acknowledged my sin unto Thee, and mine iniquity have I not hid. I said, I will confess my transgressions unto the Lord; and Thou forgavest the iniquity of my sin' (Ps. 32:5). 'Bless the Lord, O my soul, and forget not all His benefits; who forgiveth all thins iniquities' (Ps. 103:2, 3). The fact, then, is undeniable that they were justified, in the full sense of that expression,---that they were freely forgiven, and graciously accepted as righteous, so as to be restored to the favour, friendship, and fellowship of God.

  The reason or ground of their Justification was not their own personal righteousness,---for they were 'guilty,' 'ungodly,' unclean,' unable to 'stand in judgment,'---but the work of Christ, the promised Seed. For that work, although postponed till 'the fulness of times,' had a retrospective efficacy; it was accomplished for 'the redemption of the transgressions which were under the first testament' (Heb. 9:15), and Old Testament believers could say, 'He was wounded for our transgressions, and bruised for our iniquities: the chastisement of our peace was laid upon Him, and by His stripes we are healed' (Isa. 53:5). 'The covenant (of grace) was differently administered in the time of the Law, and the time of the Gospel: under the Law it was administered by promises, prophecies, sacrifices, circumcision, the paschal lamb, and other types and ordinances delivered to the people of the Jews, all fore-signifying Christ to come, which were, for that time, sufficient, and efficacious, through the operation of the Spirit, to instruct and build up the elect in faith in the promised Messiah, by whom they had full remission of sins, and eternal salvation.'---'Although the work of redemption was not actually wrought by Christ till after His incarnation, yet the virtue, efficacy, and benefits thereof, were communicated unto the elect in all ages successively from the beginning of the world, in and by those promises, types, and sacrifices, wherein He was revealed and signified to be "the Seed of the woman which should bruise the serpent's head,"---and "the Lamb slain from the beginning of the world," being "yesterday and to-day the same, and for ever."~'---Westminster Confession of Faith, c.~vii. s. 5, viii. s. 6. See Bishop Barlow, 'Remains,' pp.~584--593; Bishop O'Brien, 'Nature and Effects of Faith,' p.~439; H. Witsius, 'Animadversiones Irenicæ,' Mis. Sac. ii. p.~780; Bishop Downham 'on Justification,' p.~180.

  The means of their Justification was faith. This follows necessarily from its being left to depend on the work of Christ, for that work was still future; it was a matter of promise, and a promise can only be embraced by faith. But it is expressly declared to have been by faith; for it is written, 'The just shall live by faith' (Gal. 3:11), and 'Abraham believed God, and it was counted to him for righteousness' (Rom. 4:3; Gal. 3:6). Whether faith was itself their righteousness, and in what sense it was imputed to them, will be considered in the sequel.} And these mingled feelings of fear and hope towards God were fitly expressed, and could scarcely fail to be deepened and confirmed, by the rite of sacrifice, which formed the most solemn part of their religious worship. For that rite, as habitually practised by them, was as significant as the first promise; and its meaning was in manifest correspondence with the truths which that promise revealed. Sacrifice was offered to God in His revealed character as 'the just God,' and yet the 'Saviour of sinners;' it consisted in the slaying of an innocent animal, which was substituted in the room of the sinner, and devoted to God as an atonement for his soul, by the shedding of its blood; it implied that his sin was laid upon the head of the victim, and that his life, forfeited by sin, was redeemed by the victim's death; it expressed, on the part of every sincere worshipper, a confession of personal guilt, and a sense of penal desert, but a hope also of divine forgiveness and acceptance, for it was employed with a view to deprecate and avert God's wrath, and to implore and propitiate His favour; and the habitual observance of this rite, as the most solemn act of religious worship, had a tendency to strengthen all those feelings, both of fear and hope, of reverence and trust, of repentance and faith, which the revelation of God's justice in the curse, and of His mercy in the promise, was fitted to produce. It served also to familiarize the mind of every believer, even in primitive times, with those great principles of substitution, imputation, and vicarious satisfaction, which were involved in the divine scheme of grace and redemption, and which were only to be more fully developed, and more clearly exhibited, in connection with the person and work of the promised Seed, in 'the fulness of times.'

It has been made a question, indeed, whether the rite of sacrifice, in connection with religious worship, was an invention of man, or an institution of God. The only pretext for raising such a question arises from there being no statement in Scripture ascribing it, in express terms, to divine appointment. But apart from any categorical announcement, there may be sufficient scriptural evidence to prove that it could not have originated from the will of man, and that it must be ascribed to the revealed will of God. It is highly improbable, on the one hand, that the thought of propitiating God's favour by the slaying of His innocent animals could have suggested itself, in any circumstances, as an acceptable part of religious worship; it is still more improbable that it could have suggested itself at a time when man was not allowed to use them even for food; and it is most improbable of all that he would have ventured to introduce an act of mere will-worship into the divine service, at a time when God was revealing His mind and will, or that it would have been accepted by Him, who acted then, as He acts now, on the great principle declared in His Word,---'In vain do ye worship me, teaching for doctrines the commandments of men.' It is certain, on the other hand, that God accepted the animal sacrifice of Abel, and testified His acceptance of it, probably by fire from heaven consuming the victim on the altar,---that He accepted it in preference to the mere thank-offering of Cain, which consisted in the fruits of the ground, and had no relation to atonement by blood,---that when Cain was wroth because God had no respect to his offering, the Lord said to him, 'If thou doest well, shalt thou not be accepted? and if thou doest not well, sin lieth,' or a sin-offering coucheth, 'at the door,'---that Abel is expressly said to have offered his sacrifice in faith,'2 and faith invariably implies, according to Scripture, a divine testimony or a divine authority as its ground and warrant; and that the distinction between animals as clean and unclean,---which could have reference at that time only to sacrifice, not to food, and which depended entirely on divine appointment,---existed in the earliest times, and is repeatedly referred to in the sacred narrative. These arguments appear to me to be conclusive in favour of the divine institution of animal sacrifice as a part of solemn religious worship; but they derive additional strength from the manifest correspondence of that rite, in its spiritual significance, with the truths which had been previously revealed, and also with the method of redemption as it was subsequently more fully unfolded in the Ritual of Moses and the Gospel of Christ. For it was evidently fitted, by its radical meaning and the lessons which it taught, to be the sacrament and symbol of the first promise of a Saviour, and, as such, a type of 'the Lamb of God who should take away the sin of the world,'---a sacrament which then prefigured to the eye of faith that same sacrifice of the Cross which is now commemorated at the Lord's table. By offering that sacrifice 'in faith'---by believing the great truth which it symbolized and typified as it was revealed in the first Gospel promise,---the worshipper was justified then, as he is justified now: he obtained forgiveness and acceptance with God; and not only so, but he might enjoy the assurance of both, when, as in the case of Abel, he 'obtained witness that he was righteous, God testifying of his gifts.'\footnote{The question whether Sacrifice was a divine institution, or a human invention, has given rise to much discussion. On the one side, see Davison, 'Inquiry into the Origin and Intent of Primitive Sacrifice,' also a note in his 'Discourses on Prophecy;' 'Correspondence between Bishop Jebb and Mr.~Knox,' vol.~i. pp.~455--462; Dr.~Sykes, 'Essay on Sacrifice.' On the other, Archbishop M'Gee 'On the Atonement;' Shuckford's 'Connection of Sacred and Profane History,' vol.~i. p.~177, i. 370--385, i. 439--495, iv. pp.~48--60,---American Edition in 4 vols.; James Richie, M.D., 'Criticism on Modern Notions of Sacrifice,' particularly recommended by Dr.~M'Gee on the 'Origin of Sacrifice,' also his 'Peculiar Doctrines of Revelation,' p.~137; Dr.~John Edwards, 'Survey of Divine Dispensations,' vol.~i. 91--99; Dr.~R. Gordon, 'Christ as made known to the Ancient Church,' vol.~i. pp.~46--66; Dr.~Outram on 'Sacrifices,' passim.

  The moral meaning, and typical reference, of sacrifice, are well stated by Mr.~Beart. 'The sacrifices of old were offered in the room of the offender, whose "laying his hand thereon" (Lev. 1:4, 3:2) signified the transferring of his sin and guilt unto his victim. As if he should say, "I freely own I have deserved to die for such and such sins; but, Lord, by Thine appointment, I bring here a sacrifice, a poor animal, to die for me: accept it in my stead." It is true, these sacrifices could not do away sins (Heb. 10:1), but were referred, in their whole typical nature and use, to Christ's sacrifice, through which there is a real and eternal forgiveness, whereof that ceremonial forgiveness, which was by these sacrifices, was only a type.'---Beart's Vindication, p.~55. See Hervey's Works, ii. pp.~60, 88, 97--100, 264; P. Allinga, 'The Satisfaction of Christ,' translated by Rev.~T. Bell, Glasgow, 1790, pp.~73--90; Dr.~John Prideaux, 'Lectiones Decem,' pp.~138, 163.}

The first promise of a Saviour, commemorated and illustrated by sacrificial observances as a permanent part of divine worship, was the primæval Gospel. Both were transmitted by tradition from one generation to another, at a time when, from the longevity of men during that early age, they might long be preserved in a state of purity. That they were sufficient, under the teaching of God's Spirit, to form the characters of true believers, and to embue them with an enlightened and exalted piety, appears from the case of Abel, the first martyr for the truth, of whom it is said, that 'by faith he offered a more excellent sacrifice than Cain, by which he obtained witness that he was righteous,' or accepted as righteous in the sight of God; from the case of Enoch, of whom we read, that 'Enoch walked with God, and he was not, for God took him,' and that before his translation 'he had this testimony, that he pleased God,' or was accepted as a justified man. We have also the case of Noah, of whom it is written, that 'he found grace in the eyes of the Lord,'---that he was 'a just man, and perfect, or upright, in his generation, and walked with God,'---that he was 'a preacher of righteousness,'---and that 'he became heir of the righteousness which is by faith.'2 These cases are only specimens of primæval believers, who were justified freely by grace through faith in a promised Saviour, and who testified their faith by worshipping God, as the Holy One and the just, yet as the justifier of the ungodly,---worshipping Him in the way of His own appointment, by offering bloody sacrifices on His altar. How many they may have been, or how few, we cannot tell; but if the primæval Gospel was sufficient for the justification of all believers who worshipped God in spirit and in truth, then as long as God continued to be known in His revealed character as the just God and the Saviour, and as long as His promise---transmitted by tradition and symbolized by sacrifice---was the object of faith and hope anywhere among the children of Adam or his children's children, so long might it be, then as it now is, the 'power of God unto salvation.' For it was addressed to men universally, while as yet there was no distinction between Jew and Gentile, and no other difference betwixt man and man except the radical and permanent one, which was recognised in the first promise itself, betwixt the 'woman's seed' and 'the serpent's seed.' There was the same limit to its efficacy then as there is still, but there was no other;---all believers were justified, and none else. Unbelief was early manifested in the mere will-worship of Cain, and it gradually spread so as to become all but universal; and when 'God saw that the wickedness of man was great in the earth, and that every imagination of the thoughts of his heart was only evil continually,' He resolved to manifest, by one stupendous act of supernatural power, at once the 'curse' which He had pronounced against 'the serpent's seed,'---and the 'grace' which He had promised through 'the Seed of the woman,'---by bringing in 'the flood on the world of the ungodly,'2 and 'saving Noah and his family by a great deliverance,' that this small but precious remnant might transmit His promise, and maintain His worship, as they had received them from their believing fathers.

After the flood, the revelation of God's purpose of redeeming mercy was progressive, and became at once more copious, and more precise. In the first promise, the future Saviour had been revealed simply as 'the Seed of the woman' who should 'bruise the serpent's head;' but, as the Church advanced on her course, additional information was vouchsafed, in regard to the constitution of His person,---the line of His human descent,---the nature of the offices which He should sustain,---the work which He should accomplish,---the blessings which He should procure for His people,---and the time of His advent. That He was to be a Man, was implied in His being promised as 'the Seed of the woman;' but He was afterwards revealed to Abraham as 'the mighty God,' and at a still later period to Moses as 'Jehovah;' for it was the 'Angel of the Lord' that appeared to Moses in the bush,---who revealed Himself as 'the God of Abraham, the God of Isaac, and the God of Jacob,'---and said, 'I appeared unto Abraham, and unto Isaac, and unto Jacob, by the name of God Almighty, but by my name Jehovah I was not known unto them:' 'Thus shalt thou say unto the children of Israel, I AM hath sent me unto you.'\footnote{'The Divine Person who was so often seen by Abraham, when God was said to appear unto him, was our blessed Saviour, then in being ages before He "took upon Him the seed of Abraham." Abraham, therefore, literally speaking, saw Him; and our Saviour might very justly conclude from Abraham's thus seeing Him, that He was really in being before Abraham. Abraham built his altars, not unto God, whom "no man hath seen at any time," but unto "the Lord who appeared unto him;" and in all the accounts we have of his prayers, we find that they were offered up in the name of this Lord.'---Dr.~S. Shuckford's Connection, vol.~i. p.~177.}

In the Patriarchal age after the flood, the first and, in many respects, the most memorable case of Justification, is that of Abraham, who was to be 'the father of many nations,' and the pattern or model of all true believers till the end of time. It is frequently referred to in Scripture, not as an isolated or singular instance, having no resemblance to the justification of sinners now, but as an example or specimen which exhibits the same principles, and illustrates the same truths, that are only more clearly and fully revealed in the Gospel of the New Testament. For this reason, he is called 'the father of all them that believe;' and all believers, Christian as well as Jewish, are called 'the children of Abraham.' For the same reason, the Apostles derived some of their strongest proofs of the doctrine of Justification by grace, through faith, from that part of Scripture which records God's gracious dispensations towards him, and his experience as a sinner, who had been freely forgiven, and accepted as righteous. He was chosen and called by sovereign mercy while he was yet an idolater in the land of Chaldea. God entered into covenant with him, and called him His 'friend.'2 'The Gospel' was preached unto Abraham,---the same Gospel in substance which is now preached unto us,---even that 'in him and his seed should all the families of the earth be blessed.' By faith in that Gospel he was justified; for it is expressly recorded 'that he believed in the Lord, and He counted it to him for righteousness.'4 He believed in God, not merely as a Lawgiver, Governor, and Judge, but as 'Him who justifieth the ungodly;' and he believed in Christ as the promised 'Seed in whom all the families of the earth should be blessed,'---for, says our Lord Himself, 'Your father Abraham rejoiced to see my day; and he saw it, and was glad.'2

The Apostles made use of the case of Abraham to prove all the most important points of the doctrine of Justification. They assume that it was a case of real justification before God, declared and attested by God Himself in His inspired Word; and that it was not singular, but similar, in all essential respects, to the justification of every other sinner. They apply it to prove especially, in opposition to the prevailing opinion of the Jews, these five positions: First, that he was justified, not by works, but by faith; for 'to him that worketh, is the reward reckoned, not of grace, but of debt; but to him that worketh not, but believeth in Him that justifieth the ungodly, his faith is counted for righteousness:' Secondly, that having been justified by faith, he was consequently justified by grace; for 'therefore it is of faith, that it might be by grace;'---neither faith itself, nor any of the fruits of faith, being the ground, or the meritorious cause, of his acceptance with God: Thirdly, that having been justified by grace through faith, justification came to him, not through the Law, but through the Promise; 'for if the inheritance be of the Law, it is no more of promise, but God gave it to Abraham by promise;' but 'if they which are of the Law be heirs, faith is made void, and the promise made of none effect:' Fourthly, that having been justified by faith in God's free promise, he was not justified by circumcision or any other outward privilege: 'Cometh this blessedness, then, upon the circumcision only, or upon the uncircumcision also? for we say that faith was reckoned to Abraham for righteousness. How was it then reckoned? when he was in circumcision or in uncircumcision? Not in circumcision, but in uncircumcision; and he received the sign of circumcision, a seal of the righteousness of the faith which he had, yet being uncircumcised:' and, Fifthly, that having been justified by grace through faith in God's promise, he had no ground of boasting, or of glorying, or of self-righteous confidence; for 'if Abraham were justified by works, he hath whereof to glory, but (he had nothing whereof to glory) before God.' 'Where is boasting, then? It is excluded. By what law? of works? Nay, but by the law of faith.' These positions, deduced from the scriptural account of Abraham, will be found to exclude almost all the errors which prevailed among the Jews in the apostolic age, or which have since arisen in the Christian Church, on the subject of Justification.\footnote{On the Justification of Abraham, see Witsius, 'De Mente Pauli circa Justificationem,' Mis. Sac. vol.~ii. p.~740; Bishop Downham, 'Treatise on Justification,' pp.~317--319, 432, 486; Brown (of Wamphray), 'The Life of Justification Opened,' pp.~116, 117; Dr.~John Prideaux, 'Lectiones Decem,' p.~159; Buddeus, Misc. Sacr. vol.~ii. p.~250.}

The Patriarchs who succeeded Abraham had the same promise renewed to them, and were also justified by faith. They had peculiar privileges and hopes, as being in the direct line of the promised Seed: but there were true believers who did not belong to the family of Abraham, such as Melchizedek, 'the priest of the most high God,' and, as such, an eminent type of Christ; and 'just Lot,' 'a righteous man,' to whom 'the Lord was merciful;'4 and Abimelech, to whom the Lord revealed Himself, and acknowledged the 'integrity of his heart;' and Job, who 'was perfect and upright, one that feared God, and eschewed evil,'---who 'offered burnt-offerings' continually for his children, saying, 'It may be my sons have sinned, and cursed God in their hearts.'6 These were true believers, and, as such, accepted of God, although they were not of the seed of Abraham according to the flesh, nor directly interested in the peculiar promises of God's covenant with him; but they shared, in common with him, the first promise of a Saviour, and testified their faith in it by worshipping Jehovah in His revealed character, and offering sacrifices on His altar. Such believers were not disfranchised of their privileges or hopes by that new dispensation which first established the distinction betwixt Jews and Gentiles.\footnote{On the Theology of the Patriarchs, see J. H. Heidegger of Zurich, 'De Historia Sacra Patriarcharum, Exercitationes Selectæ,' 1667; Jurieu, 'Critical History of the Doctrines and Worship of the Church from Adam to our Saviour,' 2 vols. 8vo, translated and published at London in 1705, vol.~i. c.~1; J. T. Biddulph, 'The Theology of the early Patriarchs,' 2 vols. 8vo, 1825; and Dr.~Harris, 'Patriarchy,' a sequel to his 'Man Primeval.'}

The next great era in the History of Justification under the Old Testament was that of Moses, and the proclamation of the Law at Sinai. A new economy was now introduced, which differed in many respects from the Patriarchal system, and yet was designed and fitted, in various ways, to develop God's purpose of mercy, and to carry it on to its accomplishment in the fulness of times. That economy cannot be understood, as it is described and commented on in various parts of Scripture, unless it be contemplated in two distinct aspects: first, as a system of religion and government, designed for the immediate use of the Jews during the term of its continuance; and secondly, as a scheme of preparation for another and better economy, by which it was to be superseded when its temporary purpose had been fulfilled.

It was designed, in the first instance, for the instruction of the Jews, now formed into a nation, and about to be established in the land which the Lord had promised to give to Abraham and his seed; and, in the second instance, to prepare them, by a course of discipline and education, for the coming of Him 'in whom all the families of the earth should be blessed.' They were put 'under tutors and governors until the time appointed of the Father,'---and 'the Law was their schoolmaster to bring them unto Christ, that they might be justified by faith.' For this reason it had a mixed character,---the 'Law' which came by Moses being 'added' to the 'Promise' which had been given to Abraham. It was neither purely Evangelical, nor purely Legal; it contained the Gospel, but 'the Law was added to it because of transgressions, till the Seed should come to whom the Promise was made.'2 The addition of the Law was not intended to alter either the ground, or the method, of a sinner's justification, by substituting obedience to the Law for faith in the Promise; for the Law which was originally 'ordained unto life' was now found, by reason of sin, 'to be unto death;' but it was now 'added,' and promulgated anew with awful sanctions amidst the thunderings and lightnings of Sinai, to impress the Jews, and through them the Church at large, with a sense of the holiness and justice of Him with whom they had to do,---of the spirituality and extent of that obedience which they owed to Him,---of the number and heinousness of their sins,---and of their utter inability to escape the wrath and curse of God, otherwise than by taking refuge in the free promise of His grace. Believers were justified, therefore, under the Law, not by works, but by faith: by faith, they were 'the children of Abraham,' and 'heirs with him of the same promise.' The Law---considered as a national covenant, by which their continued possession of the land of Canaan, and of all their privileges under the Theocracy, was left to depend on their external obedience to it,---might be called a national Covenant of Works, since their temporal welfare was suspended on the condition of their continued adherence to it; but, in that aspect of it, it had no relation to the spiritual salvation of individuals, otherwise than as this might be affected by their retaining, or forfeiting, their outward privileges and means of grace. It may be considered, however, in another light, as a re-exhibition of the original Covenant of Works, for the instruction of individual Jews in the principles of divine truth; for in some such light it is evidently presented in the writings of Paul.\footnote{On the external National Covenant of the Jews, see H. Venema, 'De Fœdere Externo Veteris Testamenti,' 1771, p.~250,---being Book ii. of his Dissertations; Dr.~John Erskine (of Edinburgh), Theological Dissertations, No.~1, 1765,---'The Nature of the Sinaitic Covenant,' pp.~1--66; Bishop Warburton's 'Divine Legation of Moses,' vol.~ii. Book v. p.~235, Book vi. sec.~vi. 329; Rev.~T. Bell (of Glasgow, 1814), 'View of the Covenants of Works and Grace,' Part iv. 'The Covenant at Sinai,' p.~253; Adam Gib (of Edinburgh), 'Divine Contemplations,' c.~i.} In this aspect, it was designed, not for the justification of sinners, but for the conviction of sin. In that form, it was afterwards employed even by the Apostles of Christ, to prove the impossibility of justification by the deeds of the Law, and the necessity of another righteousness, the righteousness of faith; and for the same end, it is still applied to the conscience by every faithful preacher of the Gospel. Thus considered,---as a re-exhibition of the Covenant of Works,---it had a tendency to produce 'a spirit of bondage unto fear;' and this would have been its only effect, had it not been associated with a revelation of God's purpose and promise of grace. But when the Gospel, which had been preached beforehand to Abraham, was known and believed, so as to impart a lively apprehension of 'the forgiveness which is with God,' then conviction of sin might become genuine contrition,---remorse might be turned into repentance,---and the more thoroughly the Law had done its work in the conscience, the more gladly would the promise of a Saviour be received into the heart.

The economy of Moses, whatever prominence it gave to the Law, was unquestionably a dispensation of the Covenant of Grace. So far from superseding the promise given to Abraham, or 'making it void' and 'of none effect,' it was expressly founded upon it, and designed to carry it on to its accomplishment. That economy gathered up into itself all prior revelations of divine truth. It adopted also the Primæval and Patriarchal institutions---the Sabbath, Sacrifice, and Circumcision,---while it added to these a multitude of ordinances which were peculiar to itself---ceremonial and ritual observances, which were in themselves 'weak and beggarly elements,' and were felt to be 'a heavy yoke,'---yet they were all significant symbols, and typical prefigurations, of spiritual blessings. The believer, therefore, who could look beyond the sign to the thing signified, and see in the shadow the figure of the substance, might find Christ in every ordinance of the Old Testament Church, and obtain through Him, as revealed in the promise, forgiveness and acceptance with God. The devout Israelite, therefore, was justified by grace through faith, not less than the Christian believer. The divine Law, spiritually understood, awakened a deep conviction of sin; the divine promises, embodied and exhibited in the divine ordinances,---in those especially which related to the expiation of sin and the removal of ceremonial defilement,---pointed to a divine method of deliverance based on the principles of substitution and atonement, and produced trust in God's mercy and hope of His gracious acceptance; while the prospective character of these ordinances, as types of better things to come, and their utter insufficiency in themselves to 'take away sin,' or 'to make the comers thereunto perfect as pertaining to the conscience,' directed their thoughts forward to the time when the work of redemption should be actually accomplished by the promised Seed.\footnote{On the Justification of Old Testament believers, see Bishop O'Brien's 'Sermons on the Nature and Effects of Faith,' p.~439, 2d Edition; Witsius, 'Mis. Sac.' ii. 744, 780; Bishop Downham, 'Treatise on Justification,' p.~412; Bishop Barlow, 'Genuine Remains,' pp.~583--593; Brown (of Wamphray), 'Life of Justification,' p.~247; Dr.~John Prideaux, 'Lectiones Decem,' p.~162; Dickinson, 'Familiar Letters,' p.~191; and the precious work of Dr.~Owen on the 130th Psalm, 'works,' vol.~xiv., Russell's Edition.}

Provision was made, also, under the Law, for a growing knowledge of God's purpose and plan of redeeming mercy, by a series of Prophets, who were raised up to instruct the people in the Law, but especially to expound the promise of a Saviour, and to explain the spiritual import of the types by which He was then prefigured. Their successive announcements gave greater definiteness and precision to the meaning of both.

As Prophecy advanced, it became at once more full, and more definite, in its delineation of the person and work of the promised Saviour. It had a sudden and signal expansion in the age of David and Samuel, when the typical offices under the Law were fully established and brought into regular order. Then David began to speak of Him as 'the Christ,'---the Anointed One,---in whose person the typical offices of Prophet, Priest, and King should be combined. Afterwards Isaiah described Him as 'a man of sorrows and acquainted with grief,'---who was wounded for our transgressions, and bruised for our iniquities,'---who should 'make His soul a sacrifice for sin,' for 'the Lord hath laid upon Him the iniquities of us all,'---and then, connecting His redeeming work with the justification of His people, he adds, 'By His knowledge shall my righteous Servant justify many, for He shall bear their iniquities;' 'Surely, shall one say, In the Lord have I righteousness,'---'in the Lord shall all the house of Israel be justified and shall glory.'2 Jeremiah spoke of Him, when he said,' This is the name whereby He shall be called, The Lord our righteousness.' Zechariah spoke of Him as 'the man whose name is the Branch'---the man who is 'Jehovah's fellow,'---the 'Shepherd,'---'a Priest upon His throne;'4 and Daniel spoke of Him as 'Messiah the Prince,' who should come when the time arrived to 'anoint the Most Holy,'---'to finish the transgressions, and to make an end of sins, and to bring in everlasting righteousness.' Thus was the Gospel method of Justification 'witnessed by the Law and the Prophets,'6 for 'the testimony of Jesus was the spirit of prophecy;' and 'to Him gave all the prophets witness, that, through His name, whosoever believeth in Him shall receive remission of sins.' When He came, Moses, representing the Law, and Elijah, representing the Prophets, descended from heaven, and spake with Him 'of the decease which He should accomplish at Jerusalem;'2 and after His resurrection, 'beginning at Moses and all the prophets, He expounded in all the Scriptures the things concerning Himself.'

These truths, thus gradually revealed, were the life-blood of faith and piety in the Jewish Church; and after the time of Moses and David, when they were more fully unfolded, in connection with the office and work of the promised Seed in His character as the Messiah or the Christ, the Priesthood and the Sacrifices of the Law were regarded by every believing Israelite as 'figures' and 'types' of Him 'who should come to put away sin by the sacrifice of Himself.'\footnote{On the typical import of these rites, see Dr.~Fairbairn's 'Typology of Scripture,' 2 vols. 8vo; J. Mather on the 'Types,' as recast in 'The Gospel of the Old Testament,' 2 vols.; and Becanus, 'Analogia Veteris ac Novi Testamenti, in qua primum status Veteris, deinde Consensus, Proportio, et Conspiratio illius, cum Novo, explicatur.'} But they did not relate only to the future,---they supplied evangelical instruction to every believing Israelite; and how rich and precious that instruction was, appears from the spiritual worship which it maintained in the Church, and especially from that most marvellous record of their experience,---the Book of PSALMS. It may be safely affirmed, that every point in the Gospel doctrine of Justification is there brought out by anticipation, and strikingly exhibited in connection with the faith and worship of Old Testament believers. There is the same confession of sin: 'There is none righteous, no, not one;'---there is the same conviction of guilt and demerit: 'If Thou, Lord, shouldest mark iniquity, O Lord, who shall stand?'5---there is the same fear of God's righteous judgment: 'Visit me not in Thy wrath, chasten me not in Thy hot displeasure;'---there is the same sense of inevitable condemnation on the ground of God's Law: 'Enter not into judgment with Thy servant, for in Thy sight shall no flesh living be justified;'---there is the same earnest cry for undeserved mercy: 'Have mercy upon me, O Lord, according to Thy loving-kindness; according to the multitude of Thy tender mercies blot out my transgressions;'2---there is the same faith in His revealed character as the just God and the Saviour: 'Good and upright is the Lord; therefore will He teach sinners in the way;'---there is the same hope of pardon, resting on a propitiation; for 'with the Lord there is mercy, and with Him is plenteous redemption;'4---there is the same pleading of God's name, or the glory of all His perfections: 'For Thy name's sake, O Lord, pardon mine iniquity, for it is great;---there is the same joy and peace in believing; for 'blessed is the people that know the joyful sound: they shall walk, O Lord, in the light of Thy countenance; in Thy name shall they rejoice all the day;'6---there is the same trust in God and the faithfulness of His promises: 'I will sing of the mercies of the Lord for ever; with my mouth will I make known Thy faithfulness to all generations; for mercy shall be built up for ever, Thy faithfulness shalt Thou establish in the very heavens;'---there is the same trust in the Saviour of sinners: 'Kiss the Son, lest He be angry, and ye perish from the way: blessed are all they that put their trust in Him;'8---there is the same confidence in another righteousness than their own: 'Behold, O God, our shield, and look on the face of Thine Anointed;'---there is the same patient, persevering, hopeful waiting upon God: 'My soul, wait thou only upon God, for my expectation is from Him; He only is my rock and my salvation: He is my defence; I shall not be moved. In God is my salvation and my glory: the rock of my strength, and my refuge, is in God. Trust in Him at all times; ye people, pour out your heart before Him: God is a refuge for us.'

Every one must feel that the Old Testament, considered simply as a record of man's spiritual life and experience, stands ALONE among all the extant remains of ancient thought, and has no parallel with which it can even be compared. What is it but the Gospel, and faith in that Gospel, that gives it a character so unique, a spirit so unearthly and divine? What is it but the Gospel, pervading every page, and breathing in every utterance of contrition, or faith, or hope, that makes the book of Psalms a fit expression for the highest worship even of the Christian Church? And why, if not because the Gospel was known and believed in the Old Testament Church, and felt then, as it is felt now, to be 'the power of God unto salvation,' did the Apostles themselves seek to establish the doctrine of a free justification by grace, through faith, by making mention of the long roll of 'the elders who by faith obtained a good report,' and why did they found so much of their teaching on the recorded experience of Abraham and of David?3

Provision was thus made for the justification of sinners, by grace, through faith in the promised Saviour, throughout the whole course of the Jewish dispensation; and at its very close we find some true believers who understood its spiritual meaning,---who looked for redemption 'in Jerusalem,'---and 'waited for the consolation of Israel.' Zacharias and Elisabeth, Mary the mother of Jesus, Simeon and Anna, were ready to welcome their long-expected Saviour when He came, and gave joyful utterance to their faith in heartfelt songs of praise. It is remarkable, too, that they connected His advent with God's covenant 'promise,' and with 'the oath which He sware to their father Abraham;' for Mary, in her sublime 'Magnificat,' exclaims, 'He hath holpen His servant Israel in remembrance of His mercy, as He spake to our fathers, to Abraham, and to his seed for ever:' and Zacharias celebrates the Lord's faithfulness in fulfilling His word, 'as He spake by the mouth of His holy prophets, which have been since the world began.' These songs of faith fall on our ears like a chorus of sweet music, as the Jewish Church was ready to vanish away; and they give touching evidence of the living piety which the Old Testament still nourished within her bosom, while they form a fit introduction to the new and better dispensation of 'the fulness of times.' The Spirit of Prophecy, withdrawn since the age of Malachi, is now restored; and the Jewish Church, like an organ long silent, is once more touched by a divine hand, and its last notes resound in honour of Christ the Lord.

\end{document}
