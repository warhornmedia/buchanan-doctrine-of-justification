% Options for packages loaded elsewhere
\PassOptionsToPackage{unicode}{hyperref}
\PassOptionsToPackage{hyphens}{url}
%
\documentclass[
]{book}
\usepackage{lmodern}
\usepackage{amssymb,amsmath}
\usepackage{ifxetex,ifluatex}
\ifnum 0\ifxetex 1\fi\ifluatex 1\fi=0 % if pdftex
  \usepackage[T1]{fontenc}
  \usepackage[utf8]{inputenc}
  \usepackage{textcomp} % provide euro and other symbols
\else % if luatex or xetex
  \usepackage{unicode-math}
  \defaultfontfeatures{Scale=MatchLowercase}
  \defaultfontfeatures[\rmfamily]{Ligatures=TeX,Scale=1}
\fi
% Use upquote if available, for straight quotes in verbatim environments
\IfFileExists{upquote.sty}{\usepackage{upquote}}{}
\IfFileExists{microtype.sty}{% use microtype if available
  \usepackage[]{microtype}
  \UseMicrotypeSet[protrusion]{basicmath} % disable protrusion for tt fonts
}{}
\makeatletter
\@ifundefined{KOMAClassName}{% if non-KOMA class
  \IfFileExists{parskip.sty}{%
    \usepackage{parskip}
  }{% else
    \setlength{\parindent}{0pt}
    \setlength{\parskip}{6pt plus 2pt minus 1pt}}
}{% if KOMA class
  \KOMAoptions{parskip=half}}
\makeatother
\usepackage{xcolor}
\IfFileExists{xurl.sty}{\usepackage{xurl}}{} % add URL line breaks if available
\IfFileExists{bookmark.sty}{\usepackage{bookmark}}{\usepackage{hyperref}}
\hypersetup{
  pdftitle={The Doctrine of Justification},
  pdfauthor={James Buchanan},
  hidelinks,
  pdfcreator={LaTeX via pandoc}}
\urlstyle{same} % disable monospaced font for URLs
\usepackage{longtable,booktabs}
% Correct order of tables after \paragraph or \subparagraph
\usepackage{etoolbox}
\makeatletter
\patchcmd\longtable{\par}{\if@noskipsec\mbox{}\fi\par}{}{}
\makeatother
% Allow footnotes in longtable head/foot
\IfFileExists{footnotehyper.sty}{\usepackage{footnotehyper}}{\usepackage{footnote}}
\makesavenoteenv{longtable}
\usepackage{graphicx}
\makeatletter
\def\maxwidth{\ifdim\Gin@nat@width>\linewidth\linewidth\else\Gin@nat@width\fi}
\def\maxheight{\ifdim\Gin@nat@height>\textheight\textheight\else\Gin@nat@height\fi}
\makeatother
% Scale images if necessary, so that they will not overflow the page
% margins by default, and it is still possible to overwrite the defaults
% using explicit options in \includegraphics[width, height, ...]{}
\setkeys{Gin}{width=\maxwidth,height=\maxheight,keepaspectratio}
% Set default figure placement to htbp
\makeatletter
\def\fps@figure{htbp}
\makeatother
\setlength{\emergencystretch}{3em} % prevent overfull lines
\providecommand{\tightlist}{%
  \setlength{\itemsep}{0pt}\setlength{\parskip}{0pt}}
\setcounter{secnumdepth}{5}
% DEFINE PHYSICAL DOCUMENT SETTINGS HD
% media settings
\usepackage[paperwidth=5.5in, paperheight=8in]{geometry}

\usepackage{booktabs}
\usepackage{amsthm}
\makeatletter
\def\thm@space@setup{%
  \thm@preskip=8pt plus 2pt minus 4pt
  \thm@postskip=\thm@preskip
}

\usepackage{titling}
\usepackage{pdfpages}
\IfFileExists{./cover.pdf}{
  \newcommand{\myCover}{./cover.pdf}}
  {\IfFileExists{./cover.jpg}{
    \newcommand{\myCover}{./cover.jpg}}
    {\IfFileExists{./cover.png}{
      \newcommand{\myCover}{./cover.png}}{}
    }
  }
\@ifundefined{myCover}
{}
{
\pretitle{\begin{center}\includepdf{\myCover}}
\posttitle{\end{center}\setcounter{page}{0}}
\usepackage{atbegshi}% http://ctan.org/pkg/atbegshi
\AtBeginDocument{\AtBeginShipoutNext{\AtBeginShipoutDiscard}}
}
\clearpage\pagenumbering{roman}

\newenvironment{poetry}[0]{\par\leftskip=2em\rightskip=2em}{\par\medskip}

\newfontfamily\greekfont[Script=Greek]{LiberationSerif}

\makeatother

\frontmatter
\ifluatex
  \usepackage{selnolig}  % disable illegal ligatures
\fi
\usepackage[]{natbib}
\bibliographystyle{plainnat}

\title{The Doctrine of Justification}
\author{James Buchanan}
\date{2020}

\begin{document}
\maketitle

\mainmatter
\pagenumbering{roman}

{
\setcounter{tocdepth}{1}
\tableofcontents
}
\hypertarget{about-this-book}{%
\chapter*{About this book}\label{about-this-book}}
\addcontentsline{toc}{chapter}{About this book}

Republished by \href{https://classics.warhornmedia.com/}{Warhorn Classics}

Making classic Christian content available for free online in high quality, readable formats.

The latest version of this book can always be found \href{https://warhornmedia.github.io/warhorn-classics-book-template/}{here} in many electronic formats for your reading convenience on any device.

\hypertarget{downloads}{%
\subsubsection*{Downloads}\label{downloads}}
\addcontentsline{toc}{subsubsection}{Downloads}

\href{https://warhornmedia.github.io/warhorn-classics-book-template//Buchanan-The_Doctrine_of_Justification.pdf}{Download PDF}

\href{https://warhornmedia.github.io/warhorn-classics-book-template//Buchanan-The_Doctrine_of_Justification.epub}{Download ePub}

We hope this book is a blessing to you. If it is, please \href{https://warhornmedia.com/give}{make a one-time or recurring contribution} right now, sponsor a book from our upcoming list, or volunteer your proofreading or technical skills to help produce more content. Contact \href{mailto:lucas@beggarsborn.com}{Lucas Weeks} to get involved.

God bless,

---The Warhorn Team

\clearpage
\setcounter{page}{1}\pagenumbering{arabic}

\hypertarget{introduction}{%
\chapter{Introduction}\label{introduction}}

It may be thought by some that the subject of Justification is trite and exhausted; that, as one of the 'commonplaces' of Theology, it was conclusively determined and settled at the era of the Reformation; and that nothing new or interesting can now be introduced into the discussion of it. It is not necessary to say in reply to this, as some might be disposed to say, that 'what is new in Theology is not true, and what is true is not new;' for we believe, and are warranted by the whole history of the Church in believing, that Theology, like every other science, is progressive,---progressive, not in the sense of adding anything to the truth once for all revealed in the inspired Word, but in the way of eliciting and unfolding what has always been contained in it,---of bringing out one lesson after another, and placing each of them in a clearer and stronger light,---and discovering the connection, interdependency, and harmony, of all the constituent parts of the marvellous scheme of Revelation. In this sense, Science and Theology are both progressive, the one in the study of God's works, the other in the study of God's Word; and as human Science has not yet exhausted the volume of Nature, or reached the limit of possible discovery in regard to it, much less has human Theology fathomed the depths of Scripture, or left nothing to reward further inquiry into 'the manifold wisdom of God.' There may be room, therefore, for something new, if not in the substance, yet in the treatment, even of the great doctrine of Justification,---in the exposition of its scriptural meaning, and in the method of adducing, arranging, and applying the array of its scriptural proofs.

But apart from this, and looking to the character of our current literature, may it not be said that, to a large class of minds in the present age, nothing could well be more \emph{new} than the \emph{old} Theology of the Reformation? The Gospel is older than Luther; but, to every succeeding generation, it is still new,---good news from God,---as fresh now as when it first sprung from the fountain of Inspiration. It was new to ourselves,---surprising, startling, and affecting us strangely, as if it were almost too good to be true,---when it first shone, like a beam of heaven's own light, into our dark and troubled spirits, and shed abroad 'a peace which passeth all understanding.' It will be equally new to our children, and our children's children, when they come to know that they have sins to be forgiven, and souls to be saved; and to the last sinner who is convinced and converted on the earth, it will still be as 'good tidings from a far country,'---as 'cold water to a thirsty soul.' It can never become old or obsolete, for this obvious reason, that while it is 'the everlasting Gospel,' and, as such, like its Author, unchangeable,---'the same yesterday, and to-day, and for ever,'---yet it comes into contact, in every succeeding age, with new minds, who are ignorant of it, but need it, and can find no peace without it; and when they receive it as 'a faithful saying, and worthy of all acceptation, that Christ came into the world to save sinners,' they will learn from their own experience that the old truth is still the germ of 'a new creation'---the spring of a new life, a new peace, a new hope, a new spiritual existence, to which they were utter strangers before.

There are many, even in Protestant communities, who have long been familiar with the sound of the Gospel, to whom this inward sense of it, in its application to their own souls, would be nothing less than a new spiritual revelation. The doctrine of Justification, by grace, through faith in Christ, is the old doctrine of the Reformation, and the still older doctrine of the Gospel; yet the vivid apprehension of its meaning, and the cordial reception of its truth, must be a new thing in the experience of every one, when he is first enabled to realize and to believe it. The free pardon of all sin, and a sure title to eternal life, conferred by the mere grace of God, and resting solely on the redemption and righteousness of the Lord Jesus Christ,---this, as the actual and immediate privilege of every sinner, on the instant when he begins to rely on Christ alone for salvation, as He is offered to \emph{him} individually in the Gospel,---may come home, with all the freshness of new truth, even to many who bear the Christian name; and a realizing sense of them, in the conscious experience of their own souls, will be the best safeguard against the prevailing errors of the times, and the danger to which so many are at this moment exposed, of being 'tossed about with every wind of doctrine.'

If we take a calm survey of the state of religious sentiment in the present crisis,---for it is a crisis, and a very solemn one,---we can hardly fail to observe, that the minds of many are uneasy and unsettled; that there is a wide-spread feeling of unrest and dissatisfaction; and that this feeling manifests itself mainly in two apparently opposite tendencies, which have been so strikingly developed in the present age as to constitute its most marked and characteristic features;---the one is the tendency towards Rationalism, whose final goal is a cheerless and dreary Scepticism; the other, the tendency towards Ritualism, which can only find its complete realization in the Church of Rome. We see one large class of educated men relinquishing some of the most fundamental articles of the Christian faith, as if they had no need of them for their salvation, and contenting themselves with such lessons as Reason can learn by the mere light of Nature, or at least prove by rational arguments; and we see another large class of educated men betaking themselves to forms and ceremonies, to sacramental grace and ascetic practices, to auricular confession and priestly absolution, as if they could not find, in the simple Gospel of the grace of God in Christ, enough for their soul's need, without borrowing some additions to it from the inventions of men, and even from the corruptions of Popery. Each of these tendencies is a symptom of the same radical evil---the want of true peace, and good hope through grace; for those who have listened to Christ's voice, and complied with His gracious call, 'Come unto me, and ye shall find rest unto your souls,' have an anchor, both sure and stedfast, which keeps them, amidst all the fluctuations of human opinion, from drifting with the current; and neither Scepticism nor Superstition has any charms for them. 'They have drunk of the old wine, and have no desire for the new; for they say, The old is better.' Those who yield to these opposite tendencies differ in many respects from each other; but they agree in this: they have both abandoned the old doctrine of Justification, as revealed in the Gospel, and revived at the Reformation; and that cardinal doctrine is the one truth which alone can neutralize their respective errors, just as in the times of Luther it had power to overthrow alike the speculations of the Schools, and the superstitions of the Church. They differ in being more or less convinced of sin, more or less earnest in seeking salvation, more or less sincere in professing a reverential faith in God's Word,---for the hale-hearted Rationalist contrasts unfavourably in these respects with many an anxious-minded Ritualist,---but the Gospel doctrine of Justification, expounded in all its fulness, and exhibited in connection with the great scriptural principles which it involves or implies, is the most effective instrument at once for rousing the conscience of the Rationalist out of its false security, and for relieving the conscience of the Ritualist from its slavish anxieties and fears.

The false security of the Rationalist arises, not from the knowledge and belief of Christ's Gospel, but from ignorance or disbelief in regard to the demands and sanctions of God's Law; and the doctrine of Justification, as it is taught in Scripture, is fitted to break up that false security, and to awaken every thoughtful man to a sense of his real condition in the sight of God. For, in its negative aspect, it teaches us, first of all, how we cannot be justified,---it excludes the possibility of pardon and acceptance, in the case of man fallen, on the ground of his own obedience, and insists on the necessity of a satisfaction to divine justice, such as shall be at once an adequate expression of God's infinite abhorrence of sin, and an effectual means of securing all the ends of punishment under His moral government. What the Rationalist most needs at the outset is a work of the Law on his conscience,---a clearer and more impressive apprehension of the spirituality and extent of its preceptive requirements,---a deeper sense of sin---of the fact of sin, as undeniably chargeable against himself, and, especially, of the guilt of sin, as that which exposes him to imminent and awful danger,---a realizing conviction of those threatened penalties, which are expressive of God's holy hatred of it, and His inflexible determination to punish it,---and a close and faithful application of the whole Law to himself individually, as a sinner in the sight of God, standing before His awful tribunal, and awaiting His sentence, as a righteous Judge. Without some such experience as this, he will feel little or no interest in the question of Justification, and will scarcely be able to understand what it means, or what principles are involved in it. But that doctrine, when it is scripturally stated and explained in all its fulness, is related to the Law as well as to the Gospel; and for this reason it is admirably adapted to his case, just because it brings out, and places clearly before his conscience, the great fundamental principles of man's inexcusable guilt, and God's inflexible justice; and also because, when it proceeds to unfold a scheme of grace and redemption, it never loses sight of these principles, but exhibits them, all the more impressively, as exemplified and embodied in that scheme itself, which is a divine provision for the vindication of God's Law, with a view to the free exercise of His mercy towards the guilty. Let this doctrine take effect, first of all, in its Legal aspect,---bringing the Law to bear on his conscience, convincing him of the guilt which he has incurred, and awakening a sense of the punishment which he has deserved, as a sinner in the sight of a holy and righteous God; and then, but not till then, he will be prepared to understand and appreciate it, in its Evangelical aspect, when it proclaims a free pardon, but a pardon founded on a divine propitiation,---a gracious remission, but a remission by means of a divine redemption,---a full salvation, but a salvation procured by a divine satisfaction to God's eternal justice.

The anxieties of the Ritualist, again, arise from some sense of sin, combined with a more or less earnest desire of salvation; but accompanied also with much remaining ignorance in regard to the fulness and freeness of the Gospel provision for his immediate pardon and acceptance with God, and a latent feeling that there is still something that remains to be done or suffered by himself, in the way of satisfying the justice, averting the wrath, and propitiating the favour, of his righteous Judge. He has 'a zeal for God,' but 'not according to knowledge;' and 'he goes about to establish,' at least in part, 'his own righteousness,' instead of 'submitting,' at once and altogether, 'to the righteousness of God.' Hence he has recourse to confession and penance, not merely for the mortification of sin, but for relief from a sense of unforgiven guilt; and hence, too, his zeal in almsgiving and good works, not as expressions of gratitude for grace received, but as a means of deprecating the wrath, and securing the favour, of God. There is much in his state of mind which contrasts favourably with the careless indifference of multitudes who are at ease in Zion,---who have never felt that they have sins to be forgiven, or souls to be saved,---and who are only lulled into deeper security, and case-hardened in impenitence and unbelief, by their partial knowledge even of the message of mercy in the Gospel. One must feel a deep and tender sympathy with every earnest soul, which is really convinced of its sin and danger, and struggling to obtain deliverance,---and many a Ritualist may be in this condition. What he needs is a deeper and more thorough conviction of his ruined and helpless condition as a sinner, utterly unable to expiate any of his past sins by his own sufferings, or to secure divine acceptance by anything that he either has done, or can yet do: and along with this, a clearer perception of the perfect all-sufficiency of the finished work of Christ, to secure the immediate and full justification of every sinner, on the instant when he receives and rests on Him alone for salvation. The doctrine of Justification, therefore, as it is stated and explained in Scripture, is exactly suited to his case, just as it was to that of the Jewish Ceremonialist in apostolic times, and the Romish Ritualist at the era of the Reformation; for while, in its negative aspect, it excludes from the ground of his acceptance all works, whether done after faith or before it, and thus cuts up by the roots the principle of self-righteousness in its most insidious and seductive form, it proceeds, in its positive aspect, to bring in another righteousness---emphatically called 'the righteousness of God,' and to lay it down as 'a sure foundation in Zion;'---a righteousness already wrought out,---a righteousness already accepted,---a righteousness proposed to him individually by God Himself, as the ground on which he is warranted at once to rely for his present acceptance, and his eternal welfare. As soon as he betakes himself to this ground, and begins to rest upon it alone, he will find, in his blessed experience, that it is adequate to sustain his troubled soul,---to relieve it at once from all the anxieties of unforgiven guilt,---to set it free from 'the spirit of bondage which is unto fear,'---and to impart 'joy and peace in believing;' even that 'peace which passeth all understanding'---'the very peace of God reigning in the conscience through Jesus Christ,' and that 'joy of the Lord' which will be his 'strength' in duty, and his support in trial, enabling him to 'run in the way of His commandments' when the Lord has thus 'enlarged his heart.'

It was by the doctrine of Justification by grace through faith, as by a ray of light from heaven shining into their hearts, that the Reformers, in whose souls the work of the great spiritual revival was first wrought before it took effect on the face of Europe, obtained relief from the bondage of legal fear, and entered into the liberty wherewith Christ makes His people free.\footnote{The personal experience of the Reformers throws much light on the origin, and causes, of the Reformation.

  ``The different phases of this work succeeded each other in the mind of him who was to be the instrument of it, before it was publicly accomplished in the world. The knowledge of the Reformation, as effected in the heart of Luther himself, is, in truth, the key to the Reformation of the Church. It is only by studying the work in the individual, that we can comprehend the general work.'' \emph{(D'Aubigné, History of the Reformation in Europe, 5 vols., vol.~i. p.~140.)}

  'His conscience incessantly reminded him, that religion was the one thing needful, and that his first care should be the salvation of his soul. He had learned God's hatred of sin,---he remembered the penalties that His Word denounces against the sinner,---and he asked himself tremblingly, if he were sure that he possessed the favour of God. His conscience answered, No!' \ldots{} 'One day, when he was overwhelmed with despair, an old monk entered his cell, and spoke kindly to him. Luther opened his heart to him, and acquainted him with the fears that disquieted him. The respectable old man was incapable of entering into all his doubts, as Staupitz had done; but he knew his "Credo," and he had found there something to comfort his own heart. He thought he would apply the same remedy to the young brother. Calling his attention, therefore, to the Apostles' Creed, which Luther had learnt in his early childhood at the school at Mansfeld, the old man uttered in simplicity this article,---"I believe in the forgiveness of sins." These simple words, ingenuously uttered by the pious brother at a critical moment, shed sweet consolation in the mind of Luther. "I believe," repeated he to himself on the bed of suffering, "in the remission of sins."~'---Ib. pp.~159, 187.

  'In these spiritual conflicts and inward wrestlings, how grievously he was encumbered, fighting against incredulity, error, and desperation, marvellous it is to consider, insomuch, that three days and three nights together, he lay on his bed, without meat, drink, or any sleep, labouring in soul and spirit on a certain place of St.~Paul (Rom. 3:25, 26) which was---"to show His justice,"---thinking Christ to be sent for no other end but to show forth God's justice as an executor of His law,---till at length, being answered and satisfied by the Lord touching the right meaning of these words---signifying the justice of God to be executed upon His Son, to save us from the stroke thereof,---he immediately upon the same started up from his bed, so confirmed in faith, as that nothing afterward could appal him.'---Preface to English Version of Luther's Commentary on Galatians, translated by 'certain godly learned,' 1575, p.~v.

  'His great terror was the thought of "the righteousness of God,"---by which he had been taught to understand, His inflexible severity in executing judgment against sinners. Dr.~Staupitz and the confessor explained to him, that "the righteousness of God" is not against the sinner who believes in the Lord Jesus Christ, but for him,---not against us, to condemn, but for us, to justify. "I felt very angry," he said, "at the term---'the righteousness of God;'---for, after the manner of all the teachers, I was taught to understand it in a philosophic sense---of that righteousness, by which God is just, and punisheth the guilty\ldots. At last I came to apprehend it thus---Through the Gospel is revealed the righteousness which availeth with God,---a righteousness by which God, in His mercy and compassion, justifieth us, as it is written, 'The just shall live by faith.' Straightway I felt as if I were born anew; it was as if I had found the door of paradise thrown wide open. The expression 'the righteousness of God,' which I so much hated before, became now dear and precious,---my darling and most comforting word. I see the Father---inflexible in justice, yet delighting in mercy---'just,' beyond all my terrified conscience could picture Him, He 'justifies' me a sinner."~'---Chronicles of the Schönberg-Cotta Family, pp.~159, 160;---a graphic delineation of the state of feeling which prevailed at the time of Luther.

  Many touching allusions to his personal experience occur in the writings of Luther. For example, on the subject of self-righteousness, he says, 'I have myself taught this doctrine (i.e.~"of faith, by which embracing the merits of Christ, we stand accepted before the tribunal of God") for twenty years both in my preaching and my writings; and yet the old and tenacious mire clings to me, so that I find myself wanting to come to God, bringing something in my hand, for which He should bestow His grace upon me. I cannot attain to casting myself on pure and simple faith only, and yet this is highly necessary.' Again: 'He alludes to his former views when a monk, and the desire he then felt to converse with a saint, or holy person; figuring to himself under that name a hermit, an ascetic, feeding on roots; but he had since learned, that the saint was one, who, being justified in the righteousness of Christ, went on to serve God in his proper calling,---through the Spirit to mortify the deeds of the body, and to subdue his evil affections and desires.'---Scott's Continuation of Milner's History, i. pp.~233, 239.

  'Luther became a Reformer, because, in his confessional, he had learned to know the spiritual necessities of the people; because he had compassion on the poor people, even as the Saviour had compassion upon them. It was a hearty pity for the simple and ignorant, whom he, too, saw given up to the Priests, and Pharisees, and Scribes, and cheated of the highest blessings of life; it was a deep manly sorrow over the mistaken road of salvation along which the poor misled multitude were wandering, whereby Luther was inspirited to his first half-timid attempts; whereby, as he advanced, he was strengthened to stedfast perseverance,---whereby, at length, he was raised and arrayed as the mighty champion of evangelical freedom. Luther had rushed deep into the gulf of moral corruption, which was diffused among the lay commonalty, by the Romish doctrine of Justification by works. He knew from the liveliest experience the miserable condition to which the sincerest souls, the devoutest spirits, are reduced by this doctrine. He had found an escape for himself out of this tribulation---a path leading securely to the peace of the soul---in the righteousness of faith. Therefore he could not, and would not, keep silence at that which was going on around him. The princes and priests, indeed, the learned and educated, did not need, for the most part, that he should teach them the meaning of Indulgences, but the common uneducated people urgently demanded his help. This people Luther esteemed as standing exactly on the same level---as requiring, just like all other classes, to be led to the light of a purer knowledge of salvation; he neither deemed himself too high, or the multitude too low, to devote his services to them. In this state of mind, he boldly and powerfully tore down the wall of separation, which had been built up in the course of centuries, between the clergy and the laity. The mass of the laity, who had hitherto only been considered as a helpless body, to be moulded by the priests at pleasure, and to be interceded for by the Church before God, he roused, by the doctrine of Repentance and of Justification by faith, and gave them a living principle of spiritual independence and personality, supplying them with inexhaustible materials for contemplation, in the scriptural ideas of Sin and Divine Grace; and thus, out of the despised objects of an arbitrary away, he fashioned a living organized congregation of Christians, who had become free through their faith in their Redeemer.'---Hemdeshagen, Treatise on German Protestantism. See Archdeacon Hare, 'Vindication of Luther,' p.~296.

  'His deep, irrepressible, unappeasable consciousness of sin was the primary motive of his whole public life, and of all that he did for the reformation of the Church. It was on account of this deep feeling of the inward disease in the conscience that he tore off the plasters and lenitives with which the Romish quacks were wont to lull and skin over the wounds at the surface. It was on account of this that he set his foot on the scandalous fraud of Indulgences. It was by reason of this that he saw through the utter vanity of the penances and so-called good works, by which men were idly trying to purge their consciences. He felt, as St.~Paul and Augustine felt, that the evil in man does not lie in the imperfection of his outward works, but in the corruption of his heart and will. Therefore did he insist so strongly on the frailty which clings to our very best works; and therefore did he continually urge that, if we are to be justified, it must be wholly through grace, by the righteousness of our Divine Saviour, to be received and appropriated by faith, without any admixture of the works wrought by so frail and peccable a creature.'---Archdeacon Hare, Vindication of Luther, p.~135. See also Pfizer's Life of Luther.

  The experience of Calvin was similar to that of Luther. 'The Reformation was not the fruit of abstract reasoning; it proceeded from an inward labour,---a spiritual conflict,---a victory, which the Reformers won by the sweat of their brow, or rather, of their heart\ldots. We have on a former occasion sought to discover the generative principle of the Reformation in the heart of Luther: we are now striving to discern it in the heart of Calvin.'---D'Aubigné, History of the Reformation in the Time of Calvin, vol.~i. p.~20.

  'His chamber became the theatre of struggles as fierce as those in the cell at Erfurth. Through the same tempests, both these great Reformers reached the same haven. Calvin arrived at faith by the same practical way which had led Farel and Augustine, Luther and Paul.'---Ib. i. p.~522.

  'Calvin shut himself up in his room and examined himself. "I have been taught that Thy Son has ransomed me by His death; but I have never felt in my heart the virtue of His redemption." His Popish professors spoke to him. "The highest wisdom of Christians," they said, "is to submit to the Church, and their highest dignity is the righteousness of their works." "Alas!" replied Calvin, "I am a miserable sinner." "That is true; but there is a means of obtaining mercy. It is by satisfying the justice of God. Confess your sins to a priest, and ask humbly for absolution. Blot out the memory of your offences by good works." \ldots{} Calvin went to church, fell on his knees, and confessed his sins to God's minister, asking for absolution, and humbly accepting every penance imposed upon him\ldots. "O God," he said, "I desire by my good works to blot out the remembrance of my trespasses." He performed the satisfactions prescribed by the priest; he even went beyond the task imposed upon him; and hoped that after so much labour, he would be saved. But, alas! his peace was not of long duration\ldots. "Every time I descend into the depths of my heart---every time, O God, I lift up my soul to Thy throne, extreme terror comes over me." \ldots{} His heart was troubled; it seemed to him that every word of God he found in Scripture tore off the veil, and reproached him with his trespasses. "I begin to see," he said,---"thanks to the light that has been brought me,---in what a slough of error I have hitherto been wallowing,---with how many stains I am disfigured,---and, above all, what is the eternal death that threatens me." A great trembling came over him. He paced his room, as Luther had once paced his cell at Erfurth. He uttered, he tells us, deep groans, and shed floods of tears. Terrified at the divine holiness, like a man frightened by a violent thunder-storm, he exclaimed, "O God! Thou keepest me bowed down, as if Thy bolts were falling on my head."

  'Then he fell down, exclaiming, "Poor and wretched, I throw myself on the mercy which Thou hast shown us in Christ Jesus; I enter that only harbour of Salvation." He applied to the study of Scripture, and everywhere he found Christ. "O Father," he said, "His sacrifice has appeased Thy wrath; His blood has washed away my impurities; His Cross has borne my curse; His death hath atoned for me\ldots. Thou hast placed Thy Word before me like a torch, and Thou hast touched my heart, in order that I should hold in abomination all other merits save that of Jesus." Calvin's conversion had been long and slowly ripening; and yet, in one sense, the change was instantaneous. "When I was the obstinate slave of the superstitions of Popery," he says, "and it seemed impossible to drag me out of the deep mire, God by a sudden conversion subdued me, and made my heart obedient to His Word."~'---Ib. vol.~i. pp.~525--530.{]}} It was by the fearless proclamation of the same doctrine that they were enabled to impart immediate peace and comfort to many anxious inquirers, even in the cells and cloisters of the Church of Rome, who were prepared for its reception by those convictions of sin which the Law of God had power to awaken, but which all the Ritualism of Popery could not appease. And it was mainly to the influence of this one truth, carried home to the conscience 'in demonstration of the Spirit and with power,' that they ascribed their success, under God, in sweeping away the whole host of scholastic errors and superstitious practices, by which, in the course of many preceding centuries, men had corrupted the simpler faith and worship of the primitive Church. 'At the beginning of our preaching,' says Luther, 'the doctrine of Faith had a most happy course, and down fell the Pope's pardons, purgatory, vows, masses, and such like abominations, which drew with them the ruin of all Popery\ldots.. And if all had continued, as they began, to teach and diligently urge the article of Justification---that is to say, that we are justified neither by the righteousness of the Law, nor by our own righteousness, but only by faith in Jesus Christ,---doubtless this one article, by little and little, had overthrown the whole Papacy.'\footnote{Luther on the Epistle to the Galatians, English Translation (A.D. 1575), pp.~175, 176. Another testimony, equally clear and strong, may be quoted from the same work; for although it abounds in bold, and sometimes unguarded, statements, and is neither a learned nor a critical exposition of the Epistle, yet as a popular statement of Gospel truth, delivered first in the pulpit, and designed for the instruction of his congregation at Wittemberg, it is one of the noblest and freshest utterances which ever proceeded from the heart of a Christian divine. Mr.~Ward ventured to say of it in his 'Ideal of a Christian Church' (p.~172), that 'the Commentary, considered intellectually, as a theological effort, is perhaps one of the feeblest and most worthless productions ever written;' but those who have considered Archdeacon Hare's estimate of Mr.~Ward's competency to sit in judgment upon it, will probably attach more weight to the testimony of John Bunyan, who says of it, 'I do prefer this book of M. Luther on the Galatians, excepting the Holy Bible, before all the books that ever I have seen, as most fit for a wounded conscience.'---Hare's Vindication of Luther, 2d Ed. p.~155.

  Luther sets the doctrine of Justification by the blood of Christ through faith, against all the inventions of men, in the following striking terms:---

  'These words,---"the Son of God loved me, and gave Himself for me,"---are mighty thunderings and lightnings from heaven against the righteousness of the Law, and all the works thereof\ldots. What wilt thou do, when thou hearest the Apostle say, that such an inestimable price was given for thee? Wilt thou bring thy cowl, thy shaven crown, thy chastity, thy obedience, thy poverty, thy works, thy merits? What shall all these do? Yea, what shall the law of Moses avail? What shall the works of all men, and all the sufferings of the martyrs, profit thee? What is the obedience of all the holy angels, in comparison of the Son of God delivered, and that most shamefully, even to the death of the Cross, so that there was no drop of His most precious blood but it was shed, and that for thy sins? If thou couldst rightly consider this incomparable price, thou shouldst hold as accursed all these ceremonies, vows, works, and merits, before grace and after, and throw them down all to hell. For it is an horrible blasphemy to imagine, that there is any work whereby thou shouldst presume to pacify God, since thou seest that there is nothing which is able to pacify Him, but this inestimable price, even the death and blood of the Son of God, a drop whereof is more precious than the whole world\ldots. If I through works or merits could have loved the Son of God, and so come unto Him, what needed He to deliver Himself for me? Hereby it appeareth how coldly the Papists handled, yea, how they utterly neglected, the Holy Scriptures, and the doctrine of Faith. For if they had considered but only these words, that it behoved the Son of God to be given for me, it had been impossible that so many monstrous sects should have sprung up amongst them. For Faith would by and bye have answered, Why dost thou choose this kind of life, this religion, this work? Dost thou this to please God, or to be justified thereby? Dost thou not hear, O wretched man, that the Son of God shed His blood for thee? Thus true faith in Christ would easily have withstood all manner of sects. Wherefore I say, as I have oftentimes said, that there is no remedy against sects, or power to resist them, but this only article of Christian Righteousness. If we lose this article, it is impossible for us to withstand any errors or sects\ldots. What mean they to brag so much of works and merits? If I, being a wretched man and a damned sinner, could be redeemed by any other price, what needed the Son of God to be given for me?'---Luther on the Galatians, English Translation, p.~138.

  'The Church had fallen because the great doctrine of Justification through faith in Christ had been lost. It was therefore necessary that this doctrine should be restored to her before she could arise. Whenever this fundamental truth should be restored, all the errors and devices which had usurped its place,---the train of saints, works, penances, masses, and indulgences,---would vanish. The moment the ONE Mediator, and His ONE Sacrifice, were acknowledged, all other mediators, and all other sacrifices, would disappear. "This article of Justification," says Luther to Brentius, "is that which forms the Church,---nourishes it,---builds it up,---preserves and defends it. It is the heel which crushes the serpent's head."~'---D'Aubigné, History of Reformation in Europe, 5 vols., vol.~i. p.~73.

  'When the Gospel lifted up its voice in the days of the Reformation, the people listened. It spoke to them---of God, Sin, Condemnation, Pardon, Everlasting Life,---in a word, of Christ. The human soul discovered that this was what it wanted; and was touched, captivated, and finally renewed.'---D'Aubigné, History of the Reformation in the Time of Calvin, vol.~ii. p.~399. See also p.~583.}

If the doctrine of Justification by grace through faith be, as it unquestionably is, the only sovereign and effectual antidote to each of the two great tendencies of the age,---the tendency to Rationalism, on the one hand, and the tendency to Ritualism, on the other,---the re-exposition of it, in a form adapted to the more recent phases of these prevailing errors, might be, at least, a new and seasonable application of the old truth to the most urgent wants of men's minds in the present day; and, as such, it might be both interesting and useful, even if the doctrine of the Reformation were universally acknowledged to be still the doctrine of the Protestant Church. But an additional reason for a renewed exhibition of that truth, which has heretofore been unanimously recognised as the distinctive principle of the Reformation, may be found in the fact, that, of late years, and within the ranks of Protestantism itself, it has been openly assailed, as having no place either in the formularies of the Church of England, or in the writings of the Christian Fathers, or even in the Word of God itself. When old truths are attacked with new weapons, they must be vindicated by new defences, adapted to meet the most recent forms of error; and this is pre-eminently the case, at the present day, with the cardinal doctrine of Justification. It is not denied by its recent assailants that it was the doctrine of the leading Reformers, or that it was unanimously adopted and professed by all the churches which they founded, whether Lutheran or Calvinistic, with one singular exception only---the Church of England,---which, it seems, is neither Lutheran nor Calvinistic, and, of course, not Protestant,---and yet not Popish,---but purely Catholic and Apostolical! It is now alleged that the Reformed doctrine is a 'novelty,' which was introduced for the first time in the sixteenth century, and which, for fifteen hundred years, had been unknown to Catholic Antiquity, or the Church Universal; and that the Anglican Establishment, having always adhered to a complex rule of faith, composed of the Scriptures as interpreted by the Fathers, is unlike all other Protestant churches in this---that she has never adopted or sanctioned this novelty as part of her authorized creed. What renders this 'sign of the times' all the more significant and ominous is the additional fact, that all these assaults on the cardinal doctrine of the Reformation, from whatever quarter they have proceeded, whether from Rationalists or from Ritualists,---and they have proceeded from both,---have invariably had one and the same aim and direction---a return, in substance, if not in form, to the corrupt doctrine of the Church of Rome. The views on this important subject, which are now openly avowed in many influential quarters, are not only essentially the same with those which were exploded, we had hoped, for ever at the Reformation, but they are supported by the same arguments and the same interpretations of Scripture which were then current in the Popish Church, and which all the great divines of England---such as Davenant, and Downhame, and Barlow, and Prideaux, and Hooker---combated and demolished, especially in that marvellous age of sound theological learning, the seventeenth century. Yet now Protestants have been found willing to re-furbish the weapons of Bellarmine and Osorio, and to direct them anew against the very stronghold of our faith.

Within the last thirty years, several writers of unquestionable ability and learning, belonging at the time to the United Church of England and Ireland, have come prominently forward as uncompromising opponents of the Protestant, and zealous advocates or apologists for the essential principle of the Popish, doctrine on this subject. The first in order was a layman, but with a bishop as his coadjutor---Mr.~Knox of Dublin,---at one time private secretary to Lord Castlereagh, then Viceroy of Ireland, and all along the friend and correspondent of Wilberforce, John Wesley, and Hannah More, whose 'Correspondence' during thirty years with Dr.~Jebb, Bishop of Limerick, and also his 'Remains,' derive their chief interest from the zeal with which he opposes the doctrine of a Forensic Justification, and seeks to substitute for it that of a Moral Justification by our own inherent righteousness; a doctrine which is identical, in its radical and distinctive principle, with that of the Church of Rome. A seasonable antidote to some of the errors, which were thus sought to be revived in the Protestant Church, was supplied by Dr.~O'Brien, now Bishop of Ossory, in a work on 'The Nature and Effects of Faith;' but it was directed, in the first instance, against the doctrine of Bishop Bull, which made our justification to rest on faith and works conjointly; and it was only in the second edition, published with many enlargements, after an interval of more than twenty years, that the special views of Mr.~Knox were fully examined and criticised. Another valuable work appeared by George Stanley Faber, partly prepared on his own spontaneous motion, and partly called forth by a personal appeal addressed to him by the Editor of the two concluding volumes of Knox's 'Remains,' that he should throw the shield of his authority over the new views, by bringing his great learning to bear on the establishment of the historical fact, asserted by Knox, that the doctrine of Forensic Justification was a novelty introduced by the Reformation, and that it had no place in the genuine remains of Catholic Antiquity. The appeal was responded to, but in a style which must have surprised and disappointed its too sanguine author; for Faber's answer is a thorough vindication of the Protestant doctrine, and the conclusion at which he arrives, in regard alike to the schemes of Bull, Knox, and Trent, is, that 'not a vestige of any one of them can be discovered in the writings of Ecclesiastical Antiquity,'---a conclusion which is considerably stronger, as it appears to me, than is either warranted by the facts of the case, or necessary for the vindication of Protestant truth. His statement of the Protestant doctrine, and his proof of its having been taught by \emph{some} of the Fathers, are highly-satisfactory; but his conclusion, as thus stated, is utterly untenable, and need not be adopted by any one who does not hold that the unanimous consent of the Fathers is necessary to verify any article of faith. Let any one read 'Ancient Christianity,' by Isaac Taylor, and he can scarcely fail to be convinced that much grievous error, affecting both the doctrine and the worship of the Church, had crept in before the close of the second century, and that it is to be found, mixed with many precious truths, in the writings of the most esteemed Fathers. Indeed, the germ of it existed even in the primitive Church. (2 Thess. 2:7; 1 John 4:3)

Dr.~J. H. Newman, in his 'Lectures on Justification,' refers cursorily to the treatises of O'Brien and Faber, but offers no formal reply to them, otherwise than by expounding and attempting to establish his own theory, which is substantially the same, in its fundamental principle, with those of Bull, Knox, and Trent, although it is intended to be a middle way between the Protestant and the Popish doctrines. It was ably answered by Dr.~James Bennett and others. Dr.~Newman was then a minister of the Church of England, and is now a priest of the Church of Rome. This is of itself a significant indication of the tendency of the views which he had promulgated in the 'Tracts for the Times;' and it is deeply instructive to learn this additional fact, which is expressly stated in his recent 'Apology,' that in early life, and at what he still believes to have been the period of his conversion, he came under the influence of 'a definite creed,' and 'received impressions which have never been effaced or obscured,'---that he learned his first lessons in 'the school of Calvin,'---that the writer who made the deepest impression on his mind, and to whom, he says, '(humanly speaking) I almost owe my soul,' was Thomas Scott, the commentator,---that he admired the writings of Romaine, and 'hung upon the lips of Daniel Wilson;' yet all this Evangelical, and even Calvinistic, teaching has resulted in his renouncing the Protestant, and preferring the Romish, doctrine of a sinner's acceptance in the sight of God.\footnote{The titles of the works mentioned in the text, and the editions of them which will be referred to, are the following:---
  'Remains of Alex. Knox, Esq.,' in 4 vols. 8vo, 1834.

  'Thirty Years' Correspondence between Bishop Jebb and Mr.~Knox,' 2 vols. 8vo, 1834.

  Bishop O'Brien, 'Essays on the Nature and Effects of Faith,' 2d Edition, 1862.

  Geo. Stanley Faber, 'The Primitive Doctrine of Justification,' 2d Edition, 1839.

  Dr.~J. H. Newman, 'Lectures on Justification,' 2d Edition, 1840.

  Dr.~James Bennett, 'Justification as Revealed in Scripture, in opposition to the Council of Trent, and Mr.~Newman's Lectures,' 8vo, 1840. Dr.~Bennett had previously published a volume entitled, 'The Theology of the Early Christian Church,' being the Eighth Series of the Congregational Lecture,---New edition, 1855,---which touches on the subject of Justification, pp.~118--132, and has a direct bearing on the question whether the Protestant doctrine is a novelty which arose in the sixteenth century.

  Griffith's 'Reply to Dr.~Newman's Lectures,' commended by Bishop Daniel Wilson, has not come into my hands. Bateman, 'Life of Bishop Wilson,' p.~357.

  Dr.~J. H. Newman, 'Apologia pro Vita Sua,' 1864.}

This is only one specimen, selected from among many which might be mentioned, of a process which has been going on extensively for years past, in certain circles of society, and which, whether it results in avowed Romanism, or stops short at some intermediate stage, indicates, with sufficient clearness, an uneasy restlessness of mind, arising partly from some sense of sin, but also from superficial views of men's guilt and helplessness as sinners, and partly from inadequate apprehensions of the nature, value, and efficacy of the remedy which is provided for them in the Gospel. Hence the necessity of expounding anew, in these critical times, and that, too, for the benefit of Evangelical Protestants themselves, the full meaning, and the scriptural proofs, of the cardinal doctrine of the Gospel,---the doctrine of a full and free Justification, by grace, through faith in Christ alone. It is true that the writings to which I have referred, may be confined, in the first instance, to the educated classes, and may not directly affect the great body of the Protestant community; but, not to speak of the inevitable influence which, in this age of general literature, minds of high culture will ever exercise on popular opinion, it must never be forgotten that there is a deeper and more fertile source of error on this subject than false teaching from without,---it has an ally and an accomplice within; for there is profound truth in the memorable saying of Robert Trail: 'There is not a minister that dealeth seriously with the souls of men, but he finds an Arminian scheme of justification in every unrenewed heart.'\footnote{Robert Traill (of London), 'A Vindication of the Protestant Doctrine of Justification,' Works, vol.~i. p.~321. Reprinted by the Free Church Committee on Cheap Publications.}

That these Lectures may be adapted to the exigencies of the present times, it is necessary to keep steadily in view the theories and speculations which have recently appeared, and to suggest such considerations as may serve to neutralize or counteract their injurious influence. But they are designed to be didactic, rather than controversial. For it has long been my firm conviction, that the only effective refutation of error is the establishment of truth. Truth is one, error is multiform; and truth, once firmly established, overthrows all the errors that either have been, or may yet be, opposed to it. He who exposes and expels an error, does well; but it will only return in another form, unless the truth has been so lodged in the heart as to shut it out for ever. The great object, therefore, should be, to expound the doctrine of Justification in its full meaning, as it is revealed in Scripture,---to illustrate the great principles which are involved or implied in it,---to adduce and apply the scriptural proofs on which it rests,---and to contrast it with such other methods of obtaining pardon and acceptance with God as men have devised for themselves; and this, with a view to two practical results: first, to direct some, whose consciences have been awakened but not appeased, to a sure ground of immediate pardon and acceptance; and secondly, to direct believers, who are still burdened with doubts and fears, to such views of the nature, grounds, and evidences of this great Gospel privilege, as may serve, under the divine blessing, to raise them to a more comfortable enjoyment of it, by adding the 'assurance of faith' and 'hope' to 'the assurance of understanding.'

\end{document}
